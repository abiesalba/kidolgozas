\documentclass[
	%parspace, % Térköz bekezdések közé / Add vertical space between paragraphs
	%noindent, % Bekezdésének első sora ne legyen behúzva / No indentation of first lines in each paragraph
	%nohyp, % Szavak sorvégi elválasztásának tiltása / No hyphenation of words
	%twoside, % Kétoldalas nyomtatás / Double sided format
	%draft, % Gyorsabb fordítás ábrák rajzolása nélkül / Quicker draft compilation without rendering images
	%final, % Teendők elrejtése / Set final to hide todos
]{elteikthesis}[2021/09/20]
\usepackage{eso-pic}
\usepackage{hyperref}
\usepackage{array,multirow}
\usepackage{longtable}
\usepackage{wrapfig}
\usepackage{hyperref}
\usepackage[explicit]{titlesec}
%FIGURES AND TABLES
\usepackage{caption}
\usepackage{tcolorbox}
\usepackage{varwidth} \tcbuselibrary{skins}
\usepackage{graphicx}						
\usepackage[section]{placeins}								
\usepackage{booktabs}
\usepackage{float}
\usepackage{lipsum}  
\renewcommand{\familydefault}{\sfdefault}
\renewcommand{\thechapter}{\Roman{chapter}}
\renewcommand{\chaptername}{tétel}
\renewcommand{\thesection}{\Alph{section}}
 \newcommand{\raisedrulefill}[2][0ex]{\leaders\hbox{\rule[#1]{1pt}{#2}}\hfill}
\titleformat{\subsection}{\Large\bfseries}{\thesubsection. }{0em}{#1\,\raisedrulefill[0.4ex]{2pt}}

\newcommand\BackgroundPic{%
	\put(0,0){%
		\parbox[b][\paperheight]{\paperwidth}{%
			\vfill
			\centering
			\includegraphics[width=\paperwidth,%
			keepaspectratio]{images/cover.png}%
			\vfill
}}}

\setcounter{chapter}{-1} 
% A dolgozat
% The document
\begin{document}
	
\AddToShipoutPicture*{\BackgroundPic}
\begin{titlepage}
	\AddToShipoutPicture*{\BackgroundPic}		
	\centering
	\begin{figure}[!h]
		\begin{minipage}{0.24\textwidth}
			\centering
			\includegraphics[width=1.0\linewidth]{images/jaschik_logo}
			\captionsetup{labelformat=empty}	
			\caption{\centering}
		\end{minipage}\hfill
		\begin{minipage}{0.75\textwidth}
			\centering			
			{\scshape\LARGE\bfseries Jaschik Álmos Művészeti Szakgimnázium és Technikum\par}
			\vspace{0.5cm}
			{\scshape\bfseries\Large Képző- és iparművészeti munkatárs\\
				- Festő\par}
		\end{minipage}
	\hrule	
		\captionsetup{labelformat=empty}
	\end{figure}	
	{\huge\bfseries Kidolgozott vizsgatételek\par}
	\vfill
	{\large Csonka Szilvia\par}
	{\large 2022, Budapest\par}
	
\end{titlepage}

% Nyelv kiválasztása
% Set document language
\documentlang{magyar}
%\documentlang{english}

% Teendők listája (final dokumentumban nincs)
% List of todos (not in the final document)
%\listoftodos[\todolabel]

% Tartalomjegyzék (kötelező)
% Table of contents (mandatory)
\tableofcontents
\cleardoublepage

% Tartalom
% Main content
\chapter{Bevezetés: Tételek} % Introduction
\label{ch:bevezetes}

\cleardoublepage

\chapter{Az ókori Egyiptom művészete} % Introduction
\label{ch:1_okri_egyiptom}

\section{Az ókori egyiptom társadalmi felépítése, hiedelemvilága, művészeti korszakai és emlékei}

\vspace{0.5cm}

\tcbox[left=0mm,right=0mm,top=0mm,bottom=0mm,boxsep=0mm,
toptitle=0.5mm,bottomtitle=0.5mm,title=\centering{A tétel adatai}]{%
	
	\begin{tabular}{| p{0.25\textwidth} | p{0.75\textwidth} |}
		
		\centering{\textbf{Tétel teljes címe}}
		&
		Mutassa be az ókori Egyiptom társadalmi felépítését, hiedelemvilágát! Ismertese az ókori egyiptomi művészet korszakait, az építészet, szobrászat és festészet ránk maradt emlékeinek jellegzetességeit!
		\\\hline
		
		\centering{\textbf{Jegyzetek}}
		&
		\begin{compactitem}
			\item Az ókori egyiptomi művészet korszakai, földrajzi, társadalmi környezete.
			\item A sír- és templomépítészet típusai, felépítése, jellmzői.
			\item A szobrászat, a festészet és tárgykultúra jellemzői és stílusjegyei.
		\end{compactitem}
\end{tabular}}\hfill

\subsection*{Földrajzi elhelyezkedés}

\begin{figure}[H]
	\centering
	\tcbox[colback=gray!85!black,
	left=0mm,right=0mm,top=0mm,bottom=0mm,boxsep=1mm,toptitle=0.5mm,bottomtitle=0.5mm,
	title=\centering{Az ókori Egyiptom a Nílus mentén feküdt}]{
		\includegraphics[width=0.9\linewidth]{images/01/egyiptom_terkep}}
	\captionsetup{labelformat=empty}
	\caption{}
\end{figure}

Egyiptom éghaljata alapvetően sivatagos, ezért a kultúra a \textbf{Nílus mentén}, annak partján 5-10 km-es szélességben, több mint 1 000 km hosszan alakult ki.

Ezen ókori civilizáció kialakulásának feltétele a folyó éltető ereje volt: a Nílus. A folyó vetés előtt, júliustól novemberig áradt, termékeny iszapot terítve szét, tápanyagban gazdaggá téve a talajt és lehetővé tette az \textbf{elárasztásos gazdálkodás}t.

A terület földrajzilag és kezdetben politikailag is két részre tagolódott. \textbf{Alsó Egyiptom} északon, a Nílus deltatorkolatánál feküdt a sík vidéken. \textbf{Felső Egyiptom} attól délebbre, a magasabban fekvő területeken, a Nílus mentén hosszan kanyargó partszakaszon helyezkedett el.

\subsection*{Társadalmi felépítés}

\begin{tcolorbox}[enhanced,colframe=gray!50!white,
	colbacktitle=gray!15!white,
	coltitle=gray!50!black,
	borderline={0.5mm}{0mm}{gray!15!white},
	borderline={0.5mm}{0mm}{gray!50!white,dashed},
	attach boxed title to top center={yshift=-2mm},
	boxed title style={boxrule=0.4pt},
	title=Az ókori Egyiptom társadalmi felépítése]{
			\includegraphics[width=1.0\linewidth]{images/01/egyiptomi_tarsadalom}}
\end{tcolorbox}

Az állam élén a korlátlan hatalommal rendelkező \textbf{fáraó} állt. Az előkelőket a származás szerinti arisztokrácia alkotta, belőlük kerültek ki a főtisztviselők. Az államigazgatás és az igazságszolgáltatás vezetője a vezér. A gazdasági élet irányítószerve a kincstár, az államgépezet működését az írnokok biztosították.

A közigazgatási egységek a nomoszok (kerületek), élükön a nomarkhoszokkal (kormányzókkal). A közrendűeket két réteg alkotta: a parasztok - a föld használata fejében terménnyel és közmunkával adóztak -, és a kézművesek. A házi rabszolgák a hadifoglyokból kerültek ki.\\

\begin{compactitem}
	
	\item\textbf{ Fáraó}: a trón betöltése általában a primogenitura elve (elsőszülött öröklése) történik.\\
	\textit{A fáraó volt a hadsereg főparancsnoka, az államigazgatás és a kincstár feje, valamenynyi templom főpapja és a legfőbb bíró. Mindezeken túl úgy gondolták, hogy rajta múlik az ország termékenysége. Ő tette a földeken az első kapavágást, és ő kezdte meg az aratást.}
	
	\item\textbf{ Papi arisztokrácia}\\
	\textit{A papok - egyiptomi felfogás szerint - csupán az uralkodót helyettesítették kultikus funkcióiban. Hisz a legfőbb pap a fáraó volt! Az egyiptomi papság két fontos funkciót látott el: az istenek kultuszának szolgálatát a templomokban és a halottakról való gondoskodást, az áldozatok bemutatását.}
	
	\item \textbf{Katonai arisztokrácia}\\
	\textit{A fáraó hatalmának alapja a felügyelete alatt álló oikosz-gazdaság és a zsoldos hadsereg volt mely akatonai arisztokrácia megszilárdulásával vált lehetővé. A katonai tisztségviselők elsősorban a társadalom tehetősebb képviselői voltak: előkelők, nagyobb földbirtokosok.}
	
	\item \textbf{ Írnokok} (tisztviselők) (nyilvántartás, adóztatás)\\
	\textit{Az egyiptomi állam legfontosabb hivatalnoka volt. Az irnokok készítették a legfontosabb feljegyzéseket, összeírásokat, a különböző vallási, orvosi, irodalmi szövegeket. Az írástudás minden hivatal elnyerésének feltétele volt. Egy-egy előkelő tucatnyi irnokot foglalkoztatott, s az írnokból akár magas méltóság is válhatott.}
	
	\item \textbf{Kézművesek}\\
	\textit{A Der-el-Medinában élő, kézműves férfiaknak és nőknek tíz napos időszakokra el kellett hagyniuk a várost és családjukat, hogy dolgozni menjenek oda, ahová a fáraó és legfőbb tanácsadói parancsolták. Rövid pihenési időszak után a munkások újabb tíz napot dolgoztak.}
	
	\item\textbf{ Közrendű szabadok}\\
	\textit{A közrendű szabadok foglalkozás szerint földművesek, kézművesek és kereskedők lehettek. A "félszabadok" az ún. királyi munkások, a földbirtokhoz kötött - főleg - parasztok.
	Fontos tény továbbá, hogy a piramisokat nem rabszolgák építették, hanem a közrendű, szabad népesség.}

	\item \textbf{Parasztok}: középítkezések az áradások idején
	
	\item \textbf{Rabszolgák}\\
	\textit{A rabszolgák döntően hadjáratok folyamán kerültek Egyiptomba. Már az i.e. 2700 kö-rül keletkezett „ palermói kő „ is beszámolt arról, hogy egy núbiai hadjárat  után 7000 foglyot hurcoltak az országba.}
\end{compactitem}

\vspace{0.5cm}

Az ókori egyiptomiak úgy hitték, hogy hajdanában nem voltak földi királyok, hanem maguk az istenek uralkodtak az ország felett. Ozirisz volt az, aki földművelésre tanította az embereket. Az ő felesége volt Ízisz, fiuk pedig Hórusz, a sólyomisten. A hór szó valójában magasröptűt, magasságot is jelentett, ezért származtatták magukat tőle az uralkodók. \textbf{Az egyiptomiak királyaikat tehát az istenek földi képviselőjének tekintették.}

\subsection*{Írásmód}

	\begin{wrapfigure}{r}{0.23\textwidth}
		\tcbox[colback=darkgray!85!black,
		left=0mm,right=0mm,top=0mm,bottom=0mm,boxsep=1mm,toptitle=0.5mm,bottomtitle=0.5mm,
		title=\centering{A rosettei kő}]{
		\includegraphics[width=1.0\linewidth]{01/rosetti_ko}
	}
	\end{wrapfigure}

	A legismertebb írásmód a \textbf{hieroglifa} (a szó görög eredetű és „szent vésetet” jelent), \textbf{elsősorban a falakra került és a kultuszhoz} [= a vallásgyakorlat, a vallásos szertartások összessége] \textbf{kapcsolódott}.
	
	Maguk az egyiptomiak írásukat a \textit{medu netjeru}, „az istenek szavai” névvel illették, ezzel is utalva a legendára, mely feltalálását Thot istennek tulajdonítja.
	A hieroglifa-írást 1822-ben fejtette meg Francois Champollion [sampolion] francia tudós, az ún. rosette-i kő alapján, amely egy olyan kőtábla volt, amin ugyanaz a szöveg három nyelven volt olvasható többek közt görög betűkkel.
	
	\begin{tcolorbox}[enhanced,colframe=gray!50!white,
		colbacktitle=gray!15!white,
		coltitle=gray!50!black,
		borderline={0.5mm}{0mm}{gray!15!white},
		borderline={0.5mm}{0mm}{gray!50!white,dashed},
		attach boxed title to top center={yshift=-2mm},
		boxed title style={boxrule=0.4pt},
		title=A hieroglif írás]{
			\includegraphics[width=1.0\linewidth]{images/01/hieroglifa}}
	\end{tcolorbox}

	Az írás leggyakoribb alapanyaga - a sírfalakon kívül - a papirusztekercs volt. A háromszög keresztmetszetű papirusznád belső rostjait kivették, hosszú csíkokra vágták, 6 napig vízbe áztatták, függőlegesen majd ezekre keresztbe egymásra helyezték a csíkokat (tulajdonképpen egy fonatot hoztak létre). Majd újabb 6 napon keresztül összepréselték, kiszárították, végül a lapokat a végüknél egymáshoz ragasztották, így jött létre az összegöngyölhető tekercsforma a könnyebb tárolhatóság érdekében.
	

\subsection*{Hiedelemvilág}

Az egyiptomiak egyes természeti jelenségekben természetfölötti erőknek a megtestesülését látták, melyet a hieroglif írásban, a művészi ábrázolásokban emberi, állati, növényi vagy akár egy tárgy alakjában mintáztak meg. 

Egy-egy isten megjelenhetett nemcsak emberi, hanem állati alakban is, sőt nagyon gyakran az emberi testet állatfejjel olvasztották egybe. Ugyancsak lényeges volt az istenek elnevezése is, ami jellemezte viselőjét: az Elrejtett, az Alkotó, a Hatalmas, stb. A vallásos hiedelmek között nagy szerepe volt a mágiának, hisz az egyiptomiak felfogása szerint egy természetfeletti erő igénybe vétele kellett mindenféle baj, veszedelem, betegség leküzdéséhez. Ilyen természetfeletti erőt tulajdonítottak a különböző varázsszövegeknek.

Elképzelésük szerint a fáraó, akit Hórusz földi helytartójának tartottak szintén varázserőt sugárzott. A fáraónak a napisten, Ré fiaként való tisztelete az Óbirodalom közepétől vált általánossá.

Az isteni erők sokféleképpen való megtestesítése mellett az egyiptomi vallás egyik legjellemzőbb részei a \textbf{napkultusz} és a \textbf{halottkultusz}. Ahogy a földi élet \textbf{Ré} (napisten) napsugarainak eredménye, úgy a halál utáni továbbélés, a fennmaradás lehetőségét az Ozirisz-hit adta, s ennek része volt a mumifikálás, amivel a halott túlvilági életét akarták biztosítani.

Az egyiptomi vallás egészén két tényező hatása vonul át: a szellemhité, mely mint lélekhit az egyiptomiak páratlanul fejlett halott- és őskultuszát táplálja; továbbá a vele párosult varázslaté, mely áthat a vallásra, átalakítja az istenek erejéről való felfogást, az isteneket aláveti a varázsszövegeknek, a halottakat mentesíti az erkölcsi felelősség alól az ítéleten, alapjává lesz a király és az egyedek istenné válásának.

A vallás gyakorlásának helyei a templomok voltak, melyek egyben iskoláikkal, könyvtáraikkal a tudománynak, a művelődésnek és a teológiai irodalomnak is a centrumaiként szolgáltak.

\subsubsection*{Többistenhit, fontosabb istenek}

\begin{compactitem}
	\item \textbf{Ré} (vagy Rá, kérdéses): napisten, az emberiség atyja.
	\item \textbf{Hórusz}: az istenek királya a Földön. A mitológiában Ozirisz és Ízisz fia, és az ég ura.
	\item \textbf{Ízisz}: anyaistennő, az Ókori Egyiptom egyik leghíresebb istennője, a varázslás, a termékenység, a víz és szél, a tengerhajózás istennője, a nőiesség és a hűség szimbóluma, Ozirisz felesége, Hórusz anyja.
	\item \textbf{Ozirisz}: a túlvilág birodalmának királya és bírája. Az egyik főisten. Ízisszel és Hórusszal alkot istenháromságot. A túlvilág és a halottak istene, az alvilág életet adó ura, a termőföld istene, ő tanította meg az embert a földművelésre.
	\item Anubisz: az alvilág és holtak oltalmazója, és a bebalzsamozás istene. Fekvő sakál- vagy vadkutyaként ábrázolták, illetve sakál- vagy kutyafejű emberként. Anubisz bíraként van jelen a szív megmérésénél, amikor is a halott szívét egy mérlegre teszik, a másik serpenyőbe pedig Maat-nak, az igazság istenének tollát rakják. Ha a szív nehezebb, akkor azt felfalja Ammut, egy alvilági szörny. Ő tartja számon a holtak szívét.
	\item Széth: a vihar és káosz istene.
	\item Thot (vagy Dzsehuti): a bölcsesség és a Hold istene. Íbiszként vagy kutyafejű páviánként ábrázolták. A civilizációt hozta el az embereknek.
	\item Szobek: az istenek őrzője, krokodilisten.
	\item Hathor, az Égi Tehén: Azonosult Básztettel, a termékenység és a szerelem istennője, tehén képében jelenik meg, Íziszhez hasonlították, bár a két istennő egymástől teljesen különbözött.
	\item Ámon: az istenek királya, ugyanaz mint Ámon-Ré, a napisten.
\end{compactitem}

\vspace{0.5cm}

\begin{tcolorbox}[enhanced,colframe=gray!50!white,
	colbacktitle=gray!15!white,
	coltitle=gray!50!black,
	borderline={0.5mm}{0mm}{gray!15!white},
	borderline={0.5mm}{0mm}{gray!50!white,dashed},
	attach boxed title to top center={yshift=-2mm},
	boxed title style={boxrule=0.4pt},
	title=Az ókori Egyiptom istenei]{
		\includegraphics[width=1.0\linewidth]{images/01/istenek}}
\end{tcolorbox}

\subsection*{Halottkultusz, temetkezési szokások}

	Az egyiptomi hitvilág szerint \textbf{az embernek két lelke volt}:\textbf{ az egyik a Ba nevű test-lélek} (tulajdonképpen maga a test), \textbf{a másik a Ka nevű szellemi lélek} (életadó energia, a test állandó tanácsadója, lelkiismerete [ma talán pszichének neveznénk]). A szellemi lélek a vélekedés szerint azonban csak addig él, amíg a testlélek is, ezért a testet meg kellett őrizni az örökkévalóságig, a szellemi lelket pedig táplálni kellett. (A test megőrzését szolgálta a mumifikálás és a masszív sírépületek, a piramisok.) A hitvilág szerint a szellemi lélek minden éjszaka lemegy az alvilágba, majd minden reggel a nappal együtt újjáéled. Az élet tehát mindig Kelethez, a napkeltéhez, a halál mindig Nyugathoz, a naplementéhez kapcsolódott, ezért temetkeztek az egyiptomiak minden korszakban a Nílus bal, azaz nyugati partjára.
	
	\textbf{A test megőrzésének módja a mumifikálás volt.} (Az eljárás kiindulása bizonyára az volt, hogy a predinasztikus korban a száraz sivatagi homokba temetkeztek, ami kiszívta a test nedvességét, ezáltal meggátolta a hús bomlását.) A műveletet papok végezték a sírok mellett lévő halotti templomban:
	\begin{compactitem}
		\item Az elhunyt belső szerveit kivették, kiszárították és az ún. kanopusz-edényekbe helyezték.
		\item Az agyat az orron keresztül eltávolították, a testet és a koponyát gyantával, szurokkal töltötték ki.
		\item A testet tartósító sós lébe áztatták, 70 napig a napon szárították, bebalzsamozták, végül vászonszalagokkal légmentesen bepólyálták.
	\end{compactitem}

\begin{figure}[!h]
	\begin{minipage}{0.49\textwidth}
		\centering
		\tcbox[colback=darkgray!85!black,
		left=0mm,right=0mm,top=0mm,bottom=0mm,boxsep=1mm,toptitle=0.5mm,bottomtitle=0.5mm,
		title=\centering{Kanópuszedények}]{
			\includegraphics[width=1.0\linewidth]{01/kanopusz}
		}
	\end{minipage}\hfill
	\begin{minipage}{0.45\textwidth}
		\centering
		\tcbox[colback=darkgray!85!black,
		left=0mm,right=0mm,top=0mm,bottom=0mm,boxsep=1mm,toptitle=0.5mm,bottomtitle=0.5mm,
		title=\centering{Múmia ábrázolása}]{
			\includegraphics[width=1.0\linewidth]{01/mumia}
		}
	\end{minipage}\hfill
	\captionsetup{labelformat=empty}
	\caption{}
\end{figure}

	

\subsection*{Művészeti és birodalmi korszakok}

	Az egyiptomi művészet nem díszítésre szolgáló vagy történeteket elmesélő művészet volt, hanem kifejezetten gyakorlati funkciót töltött be: a halottkultusz része volt. Az építészetben megjelenő formáknak, a sírfestészet, sírszobrászat témáinak és stílusának mind az egyiptomi hitvilágban, mitológiában találhatjuk meg a magyarázatát.

Egyiptom történelmének nagy részét három „birodalmi” szakaszra lehet osztani (melyek a művészet szempontjából is meghatározóak voltak): az óbirodalmi szakaszra, a középbirodalmi szakaszra és az újbirodalmi szakaszra. Ezeket rövidebb átmeneti időszakok választották el egymástól. Az „átmeneti” szó itt arra utal, hogy ezekben az időkben Egyiptom nem állt egységes politikai uralom alatt, hanem más, erősebb birodalmak uralma alá került. Az egyiptomi civilizáció alapjait jóval az Óbirodalom kora előtt a Nílus folyó partján letelepülő emberek által létrehozott öntözéses földművelés fektette le, amely a városok és specializált gazdasági tevékenységek (például a kézművesség, bányászat) kialakulásához vezetett.

\tcbox[left=0mm,right=0mm,top=0mm,bottom=0mm,boxsep=0mm,
toptitle=0.5mm,bottomtitle=0.5mm,title=\centering{Az ókori Egyiptom korszakai}]{%
	
	\begin{tabular}{| p{0.21\textwidth}|p{0.15\textwidth} | p{0.55\textwidth} |}
		
		\centering{\textbf{Óbirodalom}}
		&
		kb. Kr. e. 2600-2100
		&
		Híres fáraók: Dzsószer, Kheopsz, Kefrén, Mükerinosz.
		\\\hline
		
		\centering{\textbf{Középbirodalom}}
		&
		?
		&
		???
		\\\hline
		
		\centering{\textbf{Újbirodalom}}
		&
		?
		&
		???
		\\\hline
		
		\centering{\textbf{Kései kor}}
		&
		?
		&
		???
		\\\hline
\end{tabular}}\hfill

\subsection*{Az Óbirodalom művészete (kb. Kr.e. 2600-2100)}

Első fáraója Dzsószer volt, majd utána a leghíresebb egyiptomi uralkodók következtek: Kheopsz, majd az ő fia, Kefrén, és unokája Mükerinosz. Az országot ekkor 42 kerületre osztották, melyek élén egy-egy ún. vezír, azaz hivatalnok állt. Ennek az időszaknak a legfontosabb emlékei az említett négy fáraó piramisai Szakkarában és Gízában, a fáraók piramisai és szobrai.

\subsubsection*{Piramis-építészet}

\paragraph{Funkció} A piramisok az egyiptomi uralkodók sír-építményei voltak. Céljuk a halott fáraó mumifikált testének megőrzése volt, valamint hatalmas méreteikkel a fáraó nagyságát szimbolizálták. A piramisok tetején egy aranyból készült piramidion volt: a piramist felül lezáró kissebb piramis (15-20 méter magas). Mivel a nap fényét visszaverte, csillogásával mindenfelé jelezte a hatalmas fáraó sírjának helyét. (Ezek már nincsenek meg.)

\paragraph{Építőanyag} A piramisokat mészkőtömbökből építették, amit helyben bányásztak, és szabályos téglatest formájúra faragtak. A tömbök között semmilyen kötőanyagot nem használtak. A nagyméretű kőtömböket úgy szállították, hogy a sivatagi homokon egy iszap-utat hoztak létre, amin a kőtömb könnyedén csúsztatható volt, mindössze 6 emberre volt szükség így a szállításához. Néhány esetben - a belső kamrákban - gránitot is használtak, ami a legnehezebb és a legkeményebb kőfajta. A kemény kövek használata is kifejezi, hogy a piramisokat az örökkévalóság számára építették.

\paragraph{Építők} A piramisokat építő rabszolgákról Hérodotosz görög történetíró számolt be, ez a nézet azonban mára elavulttá vált: a helyben élő szabad, fölművelő köznép élelem fejében dolgozott a fáraónak azalatt, amíg a Nílus kiáradt. 

A Gízai piramisok mellett megtalálták a munkások városát, ami kórháztól kezdve sörfőzdén át mindennel el volt látva, ami csak a mindennapi élethez szükséges. 

Az piramis belül tömör, mindössze néhány sírkamrát és az azokhoz vezető keskeny (1m magas és széles) és meredek folyosókat, valamint szellőző folyosókat tartalmaz. Az építkezés alatt a belső termekbe kívülről tükrök segítségével vezették be a fényt. A piramisok belseje mára szinte teljesen lepusztult, festésnek, díszítésnek nyoma alig maradt. Eredetileg azonban domborművek díszítették a sírkamrák falait.

\paragraph{Sírrablók} A piramisokba a fáraó holteste mellé annak minden tárgyát és számtalan adományt helyeztek, amiket már a legkorábbi időktől kezdve fosztogattak. A kincsek védelmezésére, a sírrablók megtévesztésére a piramis oldalán gyakran egy álbejáratot nyitottak, a valódi bejáratot pedig elfedték.

\subsubsection*{A piramis formájának kialakulása}

\begin{wrapfigure}{r}{0.25\textwidth}
	\tcbox[colback=darkgray!85!black,
	left=0mm,right=0mm,top=0mm,bottom=0mm,boxsep=1mm,toptitle=0.5mm,bottomtitle=0.5mm,
	title=\centering{Masztaba}]{
		\includegraphics[width=1.0\linewidth]{01/masztaba}
	}
\end{wrapfigure}

\paragraph{1. Masztabák} 
A korai időszakban az uralkodók ún. masztabákba temetkeztek. Ezek a föld alatt lévő sírkamra fölé kerültek, lapos, négyzet alaprajzú, csonkagúla alakú, gyakran egyetlen kőtömbből álló építmények voltak. Masztabák az Óbirodalom idején is készültek, ekkor azonban már nem az uralkodó, hanem a vezír-réteg, vagy a fáraó-feleségek temetkeztek ilyen sírokba.

\begin{wrapfigure}{r}{0.35\textwidth}
	\tcbox[colback=darkgray!85!black,
	left=0mm,right=0mm,top=0mm,bottom=0mm,boxsep=1mm,toptitle=0.5mm,bottomtitle=0.5mm,
	title=\centering{Dzsószer fáraó lépcsős piramisa Szakkarában}]{
		\includegraphics[width=1.0\linewidth]{01/dzsoszer_piramis}
	}
\end{wrapfigure}

\paragraph{2. Lépcsős piramis}
Az Óbirodalom idején az uralkodó megnövekedett tekintélyét, isteni voltát kifejezendő, a lapos masztabák tetejére további szűkebb négyzet-alaprajzú csonkagúlákat helyeztek, hogy magasságát növeljék. Pl.: Dzsószer fáraó lépcsős piramisa Szakkarában.

Dzsószer fáraó piramisa a lépcsős piramisok első példája volt, tulajdonképpen több, egyre kisebb, egymásra rakott masztaba. Szakkarában (a mai Kairótól kb.20 km-re délebbre) található. Kb. 60 m magas, fél-egy méter magasságú mészkőtömbökből épült fel. Alkotója Dzsószer legelső vezírje, papja, orvosa, építésze: Imhotep volt.

\begin{wrapfigure}{r}{0.35\textwidth}
	\tcbox[colback=darkgray!85!black,
	left=0mm,right=0mm,top=0mm,bottom=0mm,boxsep=1mm,toptitle=0.5mm,bottomtitle=0.5mm,
	title=\centering{Sznofru fáraó dahsúri tört falú piramisa}]{
		\includegraphics[width=1.0\linewidth]{01/dashur_piramis}
	}
\end{wrapfigure}

\paragraph{3. Tört élű piramis}
Ez az időszak a hatalmas méretű gúla alakzat felépítésének kísérleti szakasza volt. Pl.:t ört élű és tág szögű piramisok Dahsúrban.
A Dzsószer után uralkodó fáraó (Sznefer) már a sok, kis lépcsőfokokból álló gúlaformát próbálta felépíteni. Piramisát azonban homok-alapra építette, és az építkezés közben a kőtömeg alatt megsűllyedt a talaj, a piramis éleinek szögét ezért beljebb kellett dönteni, hogy az építmény alacsonyabb, ezéltal kisebb tömegű legyen. Ennek eredménye lett a dahsúri Tört élű piramis.
A Tört élű piramis közvetlen szomszédságában Sznefer építészei - valószínű, hogy kijavítsák a hibájukat - a Tört élű piramis építésének második szakaszában alkalmazott dőlésszöggel építettek fel egy másik, már valóban a híres gúla-formát mutató sírépítményt, amit a vörös színű mészkövei miatt Vörös piramisnak neveznek.


\paragraph{4. Gúla forma}
\subparagraph{A gúla forma szimbolikája}
A cél elsősorban a magasság növelése volt, hiszen az a fáraó jelentőségét fejezte ki. Másrészt a forma iránti rokonszenv oka az volt, hogy az egyiptomi hitvilágban a gúla-forma több szimbolikus jelentést hordozott. Az egyiptomi teremtés-mítosz ilyennek írta le az ősvízből kiemelkedő őshalmot, azaz az első szárazföldet. A lépcsőzetesség kifejezte a napistenhez, Réhez vezető utat, amin a fáraó szellemi lelke felment az egekbe. A forma hasonló volt a fentről szétágazó nap sugaraihoz, azaz kifejezte Ré, a napisten védelmét az eltemetett fáraó felett. Továbbá minden ősi civilizáció szakrális épülete négyszög alaprajzra kerülő toronyszerű építmény volt - lásd a mezopotámiai zikkuratokat, vagy a maják amerikai templomait, amik hasonló formájúak. A piramis-forma kialakulásának is fontos feltétele volt, hogy ez a forma volt statikailag a legjobban megoldható az ókorban.

\subparagraph{A forma} Az építmény a tökéletességet tükröző szabályos gúla formájú: négyzet alaprajz, amihez minden oldalról a négyzet oldalaival egyenlő hosszúságú oldalú, szabályos háromszögek kapcsolódnak.

\subparagraph{A Gízai piramis-együttes}
Egyiptom legnagyobb piramisát a legidősebb fáraó, Kheopsz építtette (kb.Kr.e.2570-től). Az építkezés több, mint húsz évig tartott, már a fáraó megkoronázásakor elkezdték. A piramis 146m magas (2,3 millió mészkőtömbből építették, amik egyenként másfél méter magas és egy-két méter széles faragott téglatestek, és kb. két és fél tonnát nyomnak).

\begin{wrapfigure}{r}{0.5\textwidth}
	\tcbox[colback=darkgray!85!black,
	left=0mm,right=0mm,top=0mm,bottom=0mm,boxsep=1mm,toptitle=0.5mm,bottomtitle=0.5mm,
	title=\centering{Gízai piramisok}]{
		\includegraphics[width=1.0\linewidth]{01/gizai_piramisok}
	}
\end{wrapfigure}

Kheposz piramisa mellett három kisebb piramis is áll, amelyeket felesége és lánytestvérei számára építtetett.
Kheposz fia, Kefrén építette a kissé délebbre álló piramist, ami a fotókon középen áll, és
a három sír-építmény közül a legmagasabbnak tűnik. Ennek csak az az oka, hogy a Kheopsz-piramis szintben lejjebb áll. (Kefrén piramisa 136 m magas.) Az egyetlen, amin fennmaradtak azok a mészkőlapok, amelyekkel
4
ezeket a piramisokat lefedték, hogy a lépcsőzetességet teljesen megszüntetve a sima felületű, (egy hatalmas kristály hatását keltő) szabályos gúlaformát létrehozzák.
A harmadik, még délebbe álló piramis Mükerinosz fáraóé, a legkisebb a trióból, 62m magas.

\subsubsection*{Halotti templomok}

A piramisok mellett halotti templomokat is építettek, ahol a fáraó múmiáját készítették el, illetve a halálával istenné vált fáraónak szentelt templomok szerepét töltötték be az épületek, ahol a fáraó szellemi lelkének áldoztak. Ezek a templomok igen romos állapotban maradtak fenn, a legjobb állapotban megmaradt templom Kefrén fáraóé.

\paragraph{Kefrén halotti temploma Gízában}
A templom Kefrén piramisa előtt áll, mögötte található a Szfinx, így a piramis, a Szfinx és a templom egy tengelyt alkot. A templom hatalmas gránittömbökből épült. Fő részét az a folyosó jelenti, amin a fáraó múmiáját szállították végig. A folyosó két oldalán az istenek szobrai álltak, amiket az ablakokon beeső fény világított meg.

\subsubsection*{Fáraó szobrok}

A fáraó a politika és a vallás feje volt egy személyben. A fáraó szó arab, eredetileg a memphiszi palotát és a palotaőrséget jelentette. A fáraó neve általában egy olyan rövid mondat, ami isteni eredetére utal, pl. „Ré isten fia”

\paragraph{A fáraó uralkodói jelvényei}

	\begin{compactitem}
		\item \underline{Kettős korona:}\\
		Alsó és Felső Egyiptom feletti hatalom szimbóluma. Felső Egyiptomé a fehér - valószínű, hogy ezüstből készült - korona, ami egy csúcsos hegyes sisak volt. Alsó Egyiptomé a piros korona hátul hosszú nyúlvánnyal, elől fémszállal, a homlokán a két terület istenségeinek szimbólumaival, a kobrával és a keselyűvel.
		
		\item \underline{Menész:} a nehéz kettős koronát helyettesítő vászonkendő.
		
		\item \underline{Álszakáll:} bölcsesség, a bölcs uralkodó jelképe.
		
		\item \underline{Pásztorbot alakú jogar:} a népét védelmező uralkodó.
		
		\item \underline{Légycsapó vagy ostor:} az ellenségeit kíméletlenül leigázó uralkodó.
		
		\item \underline{Állatfarok:} a köténye felső részéről lógott le, a harci ügyességekben járatos uralkodó szimbóluma.
	\end{compactitem}

\begin{figure}[H]
	\centering
	\begin{minipage}{0.45\textwidth}
		\begin{tcolorbox}[enhanced,colframe=gray!50!white,
			colbacktitle=gray!15!white,
			coltitle=gray!50!black,
			borderline={0.5mm}{0mm}{gray!15!white},
			borderline={0.5mm}{0mm}{gray!50!white,dashed},
			attach boxed title to top center={yshift=-2mm},
			boxed title style={boxrule=0.4pt},
			title=Kettős korona]{
				\includegraphics[width=1.0\linewidth]{images/01/kettos_korona}}
		\end{tcolorbox}
	\end{minipage}
	\hfill
	\begin{minipage}{0.45\textwidth}
		\begin{tcolorbox}[enhanced,colframe=gray!50!white,
			colbacktitle=gray!15!white,
			coltitle=gray!50!black,
			borderline={0.5mm}{0mm}{gray!15!white},
			borderline={0.5mm}{0mm}{gray!50!white,dashed},
			attach boxed title to top center={yshift=-2mm},
			boxed title style={boxrule=0.4pt},
			title=Pásztorbot és korbács]{
				\includegraphics[width=1.0\linewidth]{images/01/pasztorbot_korbacs}}
		\end{tcolorbox}
	\end{minipage}
	\captionsetup{labelformat=empty}
	\caption{}
\end{figure}

\paragraph{A fáraó szobrok rendeltetése}
A fáraók szobrai nem a piramisokban álltak, hanem általában a piramisok mellett lévő halotti templomokban kaptak helyet fülkékben. Nem körüljárhatók, a hátuk gyakran egy laphoz támaszkodik, a figurák szigorúan főnézetre komponáltak.

\begin{wrapfigure}{r}{0.5\textwidth}
	\begin{tcolorbox}[enhanced,colframe=gray!50!white,
		colbacktitle=gray!15!white,
		coltitle=gray!50!black,
		borderline={0.5mm}{0mm}{gray!15!white},
		borderline={0.5mm}{0mm}{gray!50!white,dashed},
		attach boxed title to top center={yshift=-2mm},
		boxed title style={boxrule=0.4pt},
		title=Kefrén fáraó szobra]{
			\includegraphics[width=1.0\linewidth]{images/01/kefren_szobra}}
	\end{tcolorbox}
\end{wrapfigure} 

A fáraó-szobor a fáraó halála után egykori ideális testének helyettesítője volt, a visszatérő szellemi lélek számára készítették. Ez volt a jelzés, hogy a lélek tudja, hogy kinek a teste van a templom melletti piramisban. A szobron található ún. kartus - a fáraó nevét tartalmazó ovális alakú jelzés - tájékoztatott a szobor kilétéről. Mivel a visszatérés idejét nem lehetett tudni, a szobornak az örökkévalóságig fenn kellett maradnia, ezért tartós, kemény kőből, a legjobb esteben gránitból készítették. (Ezt délről szállították a Nílus deltájához.)

\paragraph{A fáraószobrok stílus-jellemzői}

A szobor egyben kifejezte azt is, hogy a fáraó halálával istenné vált, így a szobor stílusának minden eleme az isteni tökéletességet kívánja tükrözni.

A fáraóról nem portrét készítettek, hanem a fáraót, mint ereje teljében lévő, kiegyensúlyozott, bölcs, fiatal és szép, azaz ideális és isteni uralkodót örökítették meg a szobrok: nyugodt, mozdulatlan testtartás, frontális beállítás (= minden testrésze a nézővel szembe néz: sem a fej, sem a törzs, sem a lábak nem fordulnak el).

Bal lába mindig előrébb helyezett (de nem lép, mozdulatlan). Magyarázatként szolgálhat erre a testtartásra az, hogy a halálakor a fáraó ezzel fejezte ki az istenek előtt az alázatát, bűnösségét: a szívének volt szüksége támasztásra, ezért kerül mindig a bal láb előbbre.

A tömegkezelés zárt, a szobor egésze tömbszerű, a figura szorosan a combjai mellett tartja a kezét. 

Mereven előrenéző, eszményített arc és tekintet, nyugodt, érzelemmentes, ráncmentes, idealizált, fiatal arc, a fejen általában a menész-korona a kobrával és keselyűvel (Alsó- és Felső-Egyiptom isteneinek szimbólumai, itt azt fejezik ki, hogy ők védelmezik az uralkodó hatalmát a két terület felett).

Az izmokat nem túlhangsúlyozó, finoman, visszafogottan izmos, idealizáló mezítelen felsőtest, egyszerű szoknya.

A kezek ökölbe szorulnak, amiben általában a túlvilági élet kulcsa vagy templomi csörgő van. 

A fáraó-szobrok az Óbirodalom idején általában életnagyságúak, a későbbi korokban méreteik a lehetőségekhez mérten megnőnek, a fáraó nagyságát fejezik ki. (Pl. a Memnón kolosszusok 25 m magasak.)
	
\cleardoublepage

\section{Az ókor meghatározó festészeti techinkái: az egyiptomi falkép}

\begin{center}
	\begin{longtable}{ | m{0.25\textwidth} | p{0.75\textwidth} | }
		
		\hline
		\multicolumn{2}{|c|}{\textbf{A tétel adatai}}
		\\ \hline
		
		\hline
		Tétel teljes címe	
		 &
		 Melyek a meghatározó festészeti technikák az ókorban? Fejtse ki, miként alakult egy egyiptomi falkép elkészítésének munkamenete, milyen alapozást, pigmenteket és kötőanyagot használtak?
		\\ \hline
		
	\end{longtable}
\end{center}
\cleardoublepage

\chapter{Az antik görög művészet}
\label{ch:2_antik_gorog}

\section{Az antik görög kor hitvilága, korszakai és építészete}

\begin{center}
	\begin{longtable}{ | p{0.25\textwidth} | p{0.75\textwidth} | }
		
		\hline
		\multicolumn{2}{|c|}{\textbf{A tétel adatai}}
		\\ \hline
		
		\hline
		Tétel teljes címe & Mutassa be az antik görög kor hitvilágát, korszakait és építészetét! Elemezze a legismertebb épületeit, művészeti alkotásait!
		\\ \hline
		
		Jegyzetek &
		\begin{compactitem}
			\item Az ókori görög művészet és kultúra korszakai, jellemzői.
			\item A templomok felépítése, az oszloprendek jellemzői.
			\item A görög szobrászat jellegzetes vonásai a különböző korszakokban.
			\item A görög mitológia világa, a görög vázafestészet.
		\end{compactitem}
		\\\hline
		
	\end{longtable}
\end{center}

\cleardoublepage

\section{Az ókori görög kultúra festészetének hatása a római kori festészetre}

\begin{center}
	\begin{longtable}{ | p{0.25\textwidth} | p{0.75\textwidth} | }
		
		\hline
		\multicolumn{2}{|c|}{\textbf{A tétel adatai}}
		\\ \hline
		
		\hline
		Tétel teljes címe &
		Milyen módon, mértékben hatottak az ókori görög kultúra festészeti megoldásai tartalmi, formai, technikai szempontból a római kor festészetére? Milyen ókori mozaiktechnikákat ismer?
		\\ \hline
		
	\end{longtable}
\end{center}

\cleardoublepage

\chapter{Az antik Róma művészete} % Introduction
\label{ch:3_antik_roma}

\section{Az antik Róma korszakai, kultúrája, építészete, szobrászata és festészete}

\tcbox[left=0mm,right=0mm,top=0mm,bottom=0mm,boxsep=0mm,
toptitle=0.5mm,bottomtitle=0.5mm,title=\centering{A tétel adatai}]{%
	
	\begin{tabular}{| p{0.25\textwidth} | p{0.7\textwidth} |}
		\hline
		Tétel teljes címe
		& 
		Mutassa be az antik Róma korszakait, kultúráját, építészetének, szobrászatának és festészetének jellegzetésseit! Soroljon fel és jellemezzen néhány fennmaradt római építményt és alkotást!
		\\ \hline
		
		Jegyzetek &
		\begin{compactitem}
			\item Az ókori Római Birodalom társadalmi felépítése, a római művészet korszakai és jellemzői.
			\item Az építészet jellemzői, építmény-típusai, a szobrászat és festészet emlékei.
			\item Pannónia provincia római kori maradványai.
		\end{compactitem}
		\\\hline
		
	\end{tabular}}

\subsection*{Történelmi háttér}

	Róma alapításaként a Kr.e. 753-as évszámot jelöli meg a történelem. A Római Birodalom népe, a latinok mellett az etruszk és a szabin nép élt az itáliai félsziget északi részén.
	
	Kezdetben ezek közül az etruszkok voltak a legjelentősebbek. Kr.e. a VIII. sz. -ban telepedtek le a mai Toszkána területén (ma Siena, Pisa, Firenze e terület legfőbb városai). Az etruszk kultúra virágkora a Kr.e. VI.-VII.sz. volt. Az etruszkok kis városállamokat alapítottak, később Kr.e. 600 körül szövetségre léptek a környező népekkel. Így hozták létre Latium központtal, Róma fővárossal a Római Birodalom még kisméretű elődjét.
	
	A VII.sz.-tól a Kr.e. első századig terjed a királyok kora, amikor királyok uralkodnak a még Itáliánál messzebbre ki nem terjedő Római Birodalom területén.
	
	A Római Birodalom virágkora a Kr.e. első századra tehető, amikortól a császárok korát számítjuk. Ebben a periódusban nő meg igen nagy mértékben a birodalom területe: a mai Spanyol- és Franciaország, Anglia déli része, a német területek déli része, a Dunántúl, a Balkán-félsziget, a görög félsziget, Kis-Ázsia, Palesztina területe, Egyiptom és Afrika észai partvidéke tartozik hozzá.
	
	A Római Birodalom a Kr.u. V.sz.-ban ketté válik: Nyugat-Római és Kelet-Római Birodalommá. Kr.u. 476-ban a népvándorlás, az Európát elárasztó barbár népek betöréseinek következtében a Nyugat-Római Birodalom megbukik, míg a keleti rész Bizánc néven fejlődésnek indul.

\subsection*{Társadalom}

A király feladatait (hadvezér, bíró és főpap) az előkelők osztották fel maguk között. Óvintézkedésként az egyeduralom újbóli kialakulása ellen a vezető tisztségviselőket, így a consulokat is csak egy évre, és kettesével választották, hogy egymást is ellenőrizhessék. A kialakult szisztéma számos eleme a római köztársaság korát is túlélte.

A társadalom vertikális szerkezete viszonylag egyszerű volt, mivel legalábbis kezdetben csupán az előkelőkből és a tőlük függő helyzetben levő népből állt. Az alacsonyabb származású személyeknek és családoknak az egyes előkelőkhöz való rendkívül szoros kötődése, a cliensi viszony vagy klientúra különböző formákban Róma egész történelme folyamán fennmaradt.

\subsection*{Hitvilág}

Az ókori Róma fennállása alatt sok vallási kultusz befolyása alatt állt, melyekből sokat átvett. A római istenségekkel a római mitológia foglalkozik.

A túlvilágról a rómaiaknak mindvégig homályos elképzeléseik voltak, mindenesetre az ősök szellemeit mélyen tisztelték, igyekeztek kiengesztelni őket. A házi oltárokon, melyeknek papja a családfő volt, naponta áldoztak kisebb ajándékokkal, étellel, gyümölccsel, kevés borral vagy olajjal.

\tcbox[left=0mm,right=0mm,top=0mm,bottom=0mm,boxsep=0mm,
toptitle=0.5mm,bottomtitle=0.5mm,title=\centering{Római istenek}]{%
	\begin{tabular}{| p{0.25\textwidth} | p{0.25\textwidth} | p{0.45\textwidth} |}
		\hline
		\textbf{Római istenek }
		&
		\textbf{Görög istenek}
		&
		\textbf{Meghatározás}
		\\ \hline\hline
		
		Zeusz & Jupiter & a legfőbb isten
		\\ \hline
		
		Poszeidon & Neptunus & a tengerek istene
		\\ \hline
		
		Hádész & Pluto & az alvilág istene
		\\ \hline
		
		Niké & Victoria 	& a győzelem istennője
		\\ \hline
		
		Árész & Mars & a háború istene
		\\ \hline
		
		Apollón & Apollo & a jóslás és költészet istene
		\\ \hline
		
		Hesztia & Vesta & a házi tűzhely istennője
		\\ \hline
		
		Artemisz & Diana & a vadászat, a hold istennője
		\\ \hline
		
		Héra & Juno & a házastársi hűség istennője
		\\ \hline
		
		Themisz & Justitia & az igazság istennője
		\\ \hline
		
		Aphrodité & Venus & a szépség, a szerelem istennője
		\\ \hline
\end{tabular}}

\begin{center}
	\tcbox[colback=darkgray!85!black,
	left=0mm,right=0mm,top=0mm,bottom=0mm,boxsep=1mm,toptitle=0.5mm,bottomtitle=0.5mm,
	title=\centering{A római panteon második fő istentriásza: Jupiter a főhelyen, Juno és Minerva}]{
		\includegraphics[width=0.7\linewidth]{03/romai_istenek}}
\end{center}

\subsection*{Építészet}

Tiszteli a görög kultúrát, így tovább viszi a görög építészet vívmányait, díszítő és tagoló elemeit. Funkcionális, célszerűségre törekszik, logikus, jól érthető.


\subsubsection{Új építészeti találmányok}

\paragraph{Cementfalazás}
A rómaiak új falazási módot alakítanak ki, új anyagokat alkalmaznak, ez a cementfalazás. Sajátossága, hogy a kívül lévő héjszerű kő és téglarakás között, belül egy cementes kőtörmelékből falmag található. Előnye: erősebb, tartósabb, mint az egyszerű kőrakás, valamint íves felületeket könnyebben lehet kialakítani vele.

\paragraph{Boltíves-pilléres szerkezet}
A rómaiak egy forradalmian új szerkezetet, tartórendszert alakítanak ki: a boltíves-pilléres szerkezetet. A görögök az oszlopos-gerendázatos szerkezetet alkalmazták, ahol az oszlop hordta a súlyt, és a vízszintes gerenda fedte le, hidalta át a teret. A boltíves-pilléres szerkezet előnye a gerendázattal szemben az, hogy több súlyt bír el, és nagyobb távolságot is át tud ívelni, hidalni. Ha a gerendás oszlopos szerkezetnél a súlyt vagy a fesztávolságot növelték, a gerenda egyszerűen eltört, mivel a súly a gerenda közepére nehezedett! A boltíves szerkezetnél az íven a súly eloszlik, és megosztva nehézkedik a pillérekre.

\paragraph{Térlefedési módok}
cementfalazással és a boltíves-pilléres szerkezettel olyan új térlefedési módokra nyílt lehetőség, amelyek íves falat képeznek, és a boltívhez hasonlóan vezetik le a súlyt. Ezek a dongaboltozat, a keresztboltozat és a kupola.
\begin{compactitem}
	\item Dongaboltozat: félhenger formájú boltozat.
	\item Keresztboltozat: két dongaboltozat kereszteződése négyzet alaprajzon.
	\item Kupola: félgömb formájú boltozat (kör alaprajzú tér fölött).
\end{compactitem}

\begin{figure}[H]
	\centering
	\tcbox[colback=darkgray!85!black,
	left=0mm,right=0mm,top=0mm,bottom=0mm,boxsep=1mm,toptitle=0.5mm,bottomtitle=0.5mm,
	title=\centering{Dongaboltozat és keresztboltozat}]{
		\includegraphics[width=0.7\linewidth]{03/terlefedes_1}}
	\captionsetup{labelformat=empty}
	\caption{}
\end{figure}

\begin{wrapfigure}{r}{0.35\textwidth}
	\tcbox[colback=darkgray!85!black,
	left=0mm,right=0mm,top=0mm,bottom=0mm,boxsep=1mm,toptitle=0.5mm,bottomtitle=0.5mm,
	title=\centering{Kompozit oszlopfejezet}]{
		\includegraphics[width=1.0\linewidth]{03/kompozit_oszlopfejezet}
	}
\end{wrapfigure}

\paragraph{Kompozit oszlopfejezet}
A rómaiak tovább használják a görög oszloprendeket, de gyakran ún. oszlopszékre (henger vagy hasáb formájú talapzat) helyezik az oszlopokat.
A görög oszloptípusok mellett kialakul egy negyedik oszlopfejezet is: a kompozit oszlopfő, ami a ión csigavonal (voluta) és a korintoszi akantuszlevél motívumait ötvözi, arányait tekintve a görög korintoszi oszlophoz hasonló.

\clearpage

\subsubsection{A mérnöki építészet}

\begin{wrapfigure}{r}{0.45\textwidth}
	\tcbox[colback=darkgray!85!black,
	left=0mm,right=0mm,top=0mm,bottom=0mm,boxsep=1mm,toptitle=0.5mm,bottomtitle=0.5mm,
	title=\centering{Pons Mulvius híd}]{
		\includegraphics[width=1.0\linewidth]{03/pons_mivius_hid}
	}
	\tcbox[colback=darkgray!85!black,
	left=0mm,right=0mm,top=0mm,bottom=0mm,boxsep=1mm,toptitle=0.5mm,bottomtitle=0.5mm,
	title=\centering{Vízvezeték}]{
	\includegraphics[width=1.0\linewidth]{03/vizvezetek}
	}
\end{wrapfigure}

\paragraph{Utak}
A rómaiak Európa-szerte hatalmas, jól kiépített úthálózatot alakítanak ki, ami a kor infrastruktúráját jelentette.

\paragraph{Hidak}
A boltíves-pilléres szerkezettel az első nagyméretű hidakat a rómaiak építik. A leghíresebb máig álló római kori híd: Pons Mulvius Rómában.

\paragraph{Vízvezetékek}
Nagyméretű, több kilométeres hosszúságú, több szintes, boltíves-pilléres szerkezetű építmény, aminek segítségével a hegyvidékből a természetes lejtést kihasználva a városokba szállították a vizet.

\paragraph{Városépítészet}
Két várostípus volt alaprajz és kialakítás alapján.

	\subparagraph{Kolónia}
	A rómaiak által meghódított városok az ún. kolóniák: a szabálytalan alakú városfalat megerősítik, és a város úthálózatát négyzetrácsossá alakítják.
	
	\subparagraph{Castrum}
	A rómaiak által alapított városok a római katonai tábornok ( ún. „castrum”) mintájára épültek: nem csak úthálózatuk szabályos, hanem a városfalak is szabályos, derékszögű négyzet vagy téglalap alakúak. Ilyen pl: Ostia (Róma közelében található)

\paragraph{Fórum}
A római fórum a római város hosszanti elrendezésű, téglalap jellegű \textbf{főtere}, ahol oszlopcsarnokok, és templom áll. A fórum kereskedelmi, politikai célokat szolgált.

	\begin{wrapfigure}{r}{0.35\textwidth}
		\tcbox[colback=darkgray!85!black,
		left=0mm,right=0mm,top=0mm,bottom=0mm,boxsep=1mm,toptitle=0.5mm,bottomtitle=0.5mm,
		title=\centering{Forum Romanum}]{
			\includegraphics[width=1.0\linewidth]{03/forum_romanum}
		}
	\end{wrapfigure}

	\subparagraph{Forum Romanum} Rómában található, Róma leghíresebb műemlékei közé tartozik. Sokáig piactér volt, majd elkezdték bővíteni, több évszázadig, több királyi és császári dinasztia alatt gyarapodott (mindegyik császár épített hozzá valamit), ezért alakja newm követi a szabályos fórum-sémát. Körülbelül 150m hosszú terület, számos templom, politikai épület (többek közt a római szenátus gyűlésterme), diadalívek találhatók rajta.

\begin{wrapfigure}{r}{0.25\textwidth}
	\tcbox[colback=darkgray!85!black,
	left=0mm,right=0mm,top=0mm,bottom=0mm,boxsep=1mm,toptitle=0.5mm,bottomtitle=0.5mm,
	title=\centering{Diadalív}]{
		\includegraphics[width=1.0\linewidth]{03/diadaliv.jpeg}
	}
\end{wrapfigure}

\paragraph{Diadalív}
Egy vagy három kapuból álló építmény, ami eredetileg arra szolgált, hogy a diadalmenetek során megtisztítsa a győztes csatából hazatérő római katonákat a háború bűneitől. Később a meghódított területeken felállítva a Római Birodalom győzelmét és hatalmát hirdették. A diadalív felületén általában a győztes csatát vagy a felvonulást ábrázoló domborművek találhatók.

\paragraph{A római bazilika}
fórum két hosszanti oldalán található \textbf{nem vallásos} rendeltetésű középület, ami gazdasági, politikai, társadalmi célokat szolgált, piac, bíráskodás helye lehetett.
Alaprajza: hosszanti elrendezésű, az épület bejárata a fórum felől, azaz a hosszabbik oldalon nyílik

\subparagraph{Belső tere} A belsőbe állított oszlopsorok segítségével több hajó (= osztatlan belső tér) alakul ki, így a belső tágas, célja, hogy minél nagyobb számú tömeget legyen képes befogadni

\subparagraph{Szerkezete} A fő és mellékhajót oszlopsorok választják el egymástól. A mellékhajó falán nincsenek ablakok, az épület megvilágítását a főhajó kiemelkedő falán lévő ablakok szolgáltatták, itt jött be a fény. Ezt a szerkezeti felépítést bazilikális szerkezetnek nevezzük.

\paragraph{Termák}
Más néven fürdők. A rómaiak által kedvelt közösségi helyeknek egyike. A görögöktől vették át. Általában 3 medencéjük volt: hideg vizes, langyos, melegvizes, amik az íves kőboltozatokkal voltak lefedve: donga-, keresztboltozattal és kupolával. A melegvizű medencéket és a fürdő többi helyiségét a padlóban és a falakban futó agyagcsövek segítségével melegítették.

\paragraph{Teátrumok és Amfiteátrumok}

	\subparagraph{Teátrum}
	Görög színház, amit a rómaiak átvettek azzal a különbséggel, hogy nem sziklába vájták a színházat, a nézőtér súlyát nem a hegyoldal tartotta meg, hanem ők maguk építették fel az alépítményt is az íves boltozatok felhasználásával. A római teátrum is félkör alakú, mint a görög.
	
	\subparagraph{Amfiteátrum}
	Kör alakú teátrum, ilyen például a Colosseum (Róma).	

\paragraph{Colosseum}

	\subparagraph{Története}
	Eredeti neve Amfiteátrum Flávium volt, mert a Flavius császári dinasztia ideje alatt épült Kr.u. 70-80 között. Az építkezést Titus császár fejezte be. Az épület a nevét a mellette álló hatalmas, kolosszális méretű Néró császárt ábrázoló szoborról kapta. Mivel az amfiteátrum Flávium a Római Birodalom legnagyobb amfiteátruma volt (50 ezer néző befogadására volt alkalmas) méltán magáévá tette a méreteire utaló Colosseum nevet.
	
	\tcbox[colback=darkgray!85!black,
	left=0mm,right=0mm,top=0mm,bottom=0mm,boxsep=1mm,toptitle=0.5mm,bottomtitle=0.5mm,
	title=\centering{A Colosseum ma és régen}]{
		\includegraphics[width=0.49\linewidth]{03/colosseum}
		\includegraphics[width=0.49\linewidth]{03/colosseum_rekonstrukcio}
	}
	
	\subparagraph{Rendeltetés}
	Az épületben gladiátor játékok és állatviadalok zajlottak. A római császárok gyakran a győztes római hadjáratokat játszatták el a gladiátorokkal. A Colosseum padlóját szigetelni lehetett, így akár vízi csaták bemutatására is volt lehetőség.
	
	\subparagraph{Alaprajz}
	A Colosseum ellipszis alaprajzú, belső része az aréna (a Róma közelében lévő tengerpart arany színű homokjáról kapta a nevét, ezzel szórták be), a játékok helye, amit körben alépítménnyel megtámasztott nézőtér övez.
	
	\subparagraph{Szerkezet és térlefedés}
	A nézőteret íves boltozatok támasztják alá, vezetik le a súlyát. A nézőtérre a sugár irányú és körbefutó folyosókon keresztül lehetett bejutni, amelyek számára a boltozatok lefedést biztosítanak. Főleg dongaboltozattal fedték ezeket a folyosókat.
	
	\subparagraph{Homlokzat}
	Három szinten árkádok (boltív + pillér) sorolódnak egymás mellé, az e fölötti szint áttörés, nyílás nélküli fal, elsősorban az árnyékolást biztosította. Az árkádok között féloszlopok kaptak helyet: alul dór, középen ión, felül korintoszi, legfelül kompozit oszlopok. Noha az oszlopok nem vesznek részt a súly alátámasztásában, az oszloprendek sorrendje a teher levezetésének megfelelően alakult, a legnagyobb teherbíró képességgel rendelkező oszloptípust került legalulra, és az egyre kisebb terhet viselni tudok egyre feljebb. A római építészetnek az az elve nyilvánul meg ebben, hogy még a díszítőelem kialakítása is a logikát tükrözi. Ezt a homlokzati kialakítást Colosseum-motívumnak nevezzük.

\subsubsection{Templomépítészet}

A római templomépítészetben megnyilvánul a görög kultúra szeretete: a római templomok követik görög elődeiket, azokból alakulnak ki.

\paragraph{Különbségek a görög templomoktól}
\begin{compactitem}
	\item Nem magaslatra épülnek, hanem síkságra, a városba, a fórum rövidebb oldalára. Ezért a görög templomokkal ellentétben a római templomoknál kialakul a főhomlokzat: azaz elsősorban egy, a fórum felé néző homlokzatukkal érvényesülnek, az oldalsó és hátsó homlokzataik kevésbé hangsúlyosak.
	
	\item Nem veszi körbe minden oldalról lépcsős talapzat, hanem csak a főhomlokzatra vezet fel lépcső, a többi rész egy pódiumon áll.
	
	\item Főhomlokzatuk arányai a széles görög templomokkal ellentétben keskenyebbek és magasabbak.
\end{compactitem}

\paragraph{Két típusa}

	\subparagraph{Álperipterosz}
	Hosszanti elrendezésű templomtípus a görög peripterosz alaprajzból alakul ki, de míg a görög peripteroszt minden oldalról oszlopok veszik körül, a római épület oldal- és hátsó homlokzatán az oszlopok nem szabadon állnak, hanem féloszlopokká alakulnak, azaz a falhoz tapadnak.
	
	\subparagraph{Tholosz}
	Az azonos nevű, kerek, oszlopokkal körülvett görög templomtípust követi
	Vesta templom: tholosz alaprajzú templom, Korintoszi oszlopok veszik körül a falát. A vesta szüzek temploma volt. Sajátossága, hogy nincs gerendázata. A görög tholosz alaprajztól az különbözteti meg, hogy a római templomokhoz méltón pódiumon áll.

\begin{figure}[H]
	\centering
	\begin{minipage}{0.6\textwidth}
		\tcbox[colback=darkgray!85!black,
		left=0mm,right=0mm,top=0mm,bottom=0mm,boxsep=1mm,toptitle=0.5mm,bottomtitle=0.5mm,
		title=\centering{Álperipterosz}]{
			\includegraphics[width=1.0\linewidth]{03/alperipterosz}
		}
	\end{minipage}
	\hfill
	\begin{minipage}{0.35\textwidth}
		\tcbox[colback=darkgray!85!black,
		left=0mm,right=0mm,top=0mm,bottom=0mm,boxsep=1mm,toptitle=0.5mm,bottomtitle=0.5mm,
		title=\centering{Vesta templom, Róma}]{
			\includegraphics[width=1.0\linewidth]{03/vesta_templom}
		}
	\end{minipage}
\end{figure}

\paragraph{Pantheon}

	\subparagraph{Történet}
	A Pantheont a római építészet csúcsának is tartják. Kr.u. 120-130-ig épült. A Pantheon név (pan = összes, theosz = isten) az összes római istenre utal, akiknek szentelték a templomot. Az épületnek nem csak a neve, homlokzata is hasonlít az athéni Parthenónhoz: oszlopos timpanonos, de oszlopfejezetei korintosziak.
	
	\subparagraph{Alaprajz}
	2 részből áll: egy téglalap alakú előcsarnokból, melyet oszlopok több részre osztanak és egy hatalmas osztatlan kör alakú térből. Ez a belső 43 m átmérőjű volt, és térlefedésének másfél évezredig csodájára jártak, mert egészen a reneszánszig nem tudtak hasonlót építeni.

\begin{center}
	\tcbox[colback=darkgray!85!black,
	left=0mm,right=0mm,top=0mm,bottom=0mm,boxsep=1mm,toptitle=0.5mm,bottomtitle=0.5mm,
	title=\centering{Pantheon, Róma}]{
		\includegraphics[width=0.45\linewidth]{03/pantheon}
		\includegraphics[width=0.4\linewidth]{03/pantheon_belseje}
	}
\end{center}

	
	\subparagraph{Térlefedés}
	A 43 m átmérőjű osztatlan belső tér egyetlen kupolával van lefedve.
	
	\subparagraph{Szerkezet}
	Ahhoz, hogy ekkora súlyt meg tudjanak tartani a falak, a következőkre volt szükség:
	\begin{compactitem}
		\item rendkívül erős falak: 6-7 m széles falvastagság
		\item ne bontsa meg semmilyen nyílás a kupolát és a falakat: nincsenek ablakok a falakon
		\item a kupolában egymásra nehézkedő boltívek váza tartja alapvetően a súlyt
		\item amennyire csak lehet, csökkenteni kellett a kupola súlyát: ezt úgy érték el, hogy alulról fölfelé egyre kevesebb fajsúlyú köveket helyeztek, és legfelül már habkövek vannak.
	\end{compactitem}

	\subparagraph{Belső}
	A belsőben a fényt a kupola tetején lévő 9 m átmérőjű kör alakú nyílás, az opeion biztosítja, ez az épület egyetlen fényforrása. A nyíláson keresztül máig beesik a csapadék. Az épület márványburkolatos padlójában azonban kis vízelvezető tölcsérek, rések találhatók, amik hamar szárazzá teszik a felületet. A kupola belső felülete kazettás díszítésű, ami nem pusztán díszít, de szintén szerepet játszik a súlylevezetésben is. A fal középső részén timpanonos álablakok találhatók, alul szintén timpanonos keretezésű fülkék, melyekben egykor a római istenek szobrai álltak.

\subsubsection{Pompeii városa}

\begin{wrapfigure}{r}{0.45\textwidth}
	\tcbox[colback=darkgray!85!black,
	left=0mm,right=0mm,top=0mm,bottom=0mm,boxsep=1mm,toptitle=0.5mm,bottomtitle=0.5mm,
	title=\centering{Pompeii}]{
		\includegraphics[width=1.0\linewidth]{03/pompeii}
	}
	\tcbox[colback=darkgray!85!black,
	left=0mm,right=0mm,top=0mm,bottom=0mm,boxsep=1mm,toptitle=0.5mm,bottomtitle=0.5mm,
	title=\centering{Pompeii lakosai}]{
		\includegraphics[width=1.0\linewidth]{03/pompeii_lakosai}
	}
\end{wrapfigure}

Rómától délre Nápoly közelében található a város, a Vezúv (vulkán) lábánál. Kr.u. 79-ben a vulkán kitört és a várost betemette a hamu és a vulkanikus kőtörmelék, így az szinte teljesen a föld alá került, és fennmaradt. (Mellette szintén hasonlóan maradt fenn Herculáneum városa, de ez kevésbé feltárt).

Pompeii vesztét az okozta, hogy a láva közvetlenül nem fenyegette, a város lakói azonban nem számítottak arra, hogy a szél a város fölé fújja a vulkanikus gázokat. A lakókat a gázok és a levegő keveredéséből keletkezett kőtörmelék és hamu fedte be kb. 3 méter magasságban. Az emberek nem tudtak elmenekülni, és a kén és széndioxid miatt megfulladtak. Herculáneumot valóban a láva pusztította el.

Pompeii bortermeléssel és textiliparral foglalkozott. A városiak valószínűleg tudtak a veszélyről, mivel a Vezúv már 10 évvel korábban is kitört egyszer, de maradtak, mert a vulkáni talaj tökéletes bortermelő vidékké tette a környéket. Fennmaradtak textiltisztító kádak, a pékség malmai, a diadalív, a főút, 3 m magasságig a lakóépületek, freskóval festett politikai hirdetések, a színház épülete, amfiteátrum, a város fóruma és bazilikái.

\clearpage

\subsection*{A Római Birodalom szobrászata}

\subsubsection{Álltalános jellemzők}

Általában valamilyen kapcsolatban áll a politikai propagandával, célja, hogy a Római Birodalom tekintélyét, méltóságát hirdesse, növelje.

Stílusbeli újításokat, újszerűségeket alig tapasztalunk:
\begin{compactitem}
	\item A római szobrászat stílusa a görög szobrászat stílusát követi (a klasszikus és hellenizmus korit). ()Ennek oka: a hellenizmus kori görög szobrászat stílusa már Nagy Sándor korában elterjedt a mai Görögország, Kis-Ázsia és Dél-Itália területén, majd a római katonák szó szerint magukkal vitték a meghódított területetek görög istenszobrait Itáliába, hogy a diadalmenetek alkalmával felmutassák azokat a győzelem jeleként. Számos másolat készül görög szobrokról. A hellenizmus kori szobrokat általában csak római másolataikból ismerjük.)
	\item A római szobrászat sajátossága a domborművészetben, hogy a közeli figurákat magasan faragják (a fejek elválnak a faltól, és ezért mára le is törtek általában), a háttér motívumainak távlatát viszont lapos faragásmóddal érzékeltetik.
	\item A domborművészet jellemzője még a folyamatos komponálás: a történet eseményei között nincsen elválasztás, a történetet balról jobbra haladva szinte olvasni lehet, tehát elbeszélő jellegű lesz.
	\item Az épületek ábrázolása, a tér a perspektíva szabályai szerint alakul.
	\item A portrészobrászatban a realizmus érvényesül.
\end{compactitem}

A római szobrászat leggyakoribb témája az allegória = valamilyen fogalom, földrajzi hely, természeti jelenség, érzelem vizuális, képi megjelenítése általában emberi alak formájában.

A római szobrászat három fő emlékcsoportja:
\begin{compactitem}
	\item Domborművek – oltárok, diadalívek, diadaloszlopok domborművei.
	\item Portré – császárok mell- és fejszobrai.
	\item Egész alakos szobrok.
\end{compactitem}


\subsubsection{Domborművek}

	\paragraph{Augusztusz oltára}
	Augustus császár, akinek politikai jelszava a béke volt, állítatta az oltárt Pax Augustának, a béke istennőjének.
	
	A szabadban felállított, kis pódiumon álló, három oldalán falakkal körülvett oltár, melynek falain domborművek vannak:
	\begin{compactitem}
		\item Az Anyaföld (Itália) allegóriája – dombormű. Gyermekeit tápláló anyaként jelenik meg Itália földje, a két keblei felé forduló gyermek az ölében Romulusra és Rémusra, tágabban a római népre utal. Körülöttük a termékeny föld adományai: állatok, növények.
		
		\item Aeneász Itáliáért áldoz- dombormű. Az előtér alakjai magasan faragottak, a háttér motívuma – a templom – lapos faragású. A lapos faragásmóddal a templom vonásai elmosódottabbnak tűnnek.
	\end{compactitem}

\begin{figure}[H]
	\begin{minipage}{0.32\textwidth}
		\tcbox[colback=darkgray!85!black,
		left=0mm,right=0mm,top=0mm,bottom=0mm,boxsep=1mm,toptitle=0.5mm,bottomtitle=0.5mm,
		title=\centering{Augusztus oltára}]{
			\includegraphics[width=1.0\linewidth]{03/augusztus_oltar}
		}
	\end{minipage}
	\hfill
	\begin{minipage}{0.29\textwidth}
		\tcbox[colback=darkgray!85!black,
		left=0mm,right=0mm,top=0mm,bottom=0mm,boxsep=1mm,toptitle=0.5mm,bottomtitle=0.5mm,
		title=\centering{Az Anyaföld (Itália) allegóriája}]{
			\includegraphics[width=1.0\linewidth]{03/italia_dombormu}
		}	
	\end{minipage}
	\hfill
	\begin{minipage}{0.32\textwidth}
		\tcbox[colback=darkgray!85!black,
		left=0mm,right=0mm,top=0mm,bottom=0mm,boxsep=1mm,toptitle=0.5mm,bottomtitle=0.5mm,
		title=\centering{Aeneász Itáliáért áldoz}]{
			\includegraphics[width=1.0\linewidth]{03/aeneasz}
		}
	\end{minipage}		
\end{figure}
	

\paragraph{Titus diadalívének domborműve}
A jeruzsálemi hadjáratból hazatérő, a diadalmeneten felvonuló, a salamoni templom kegytárgyait bemutató (pl. menóra = hátágú gyertyatartó) katonákat ábrázolja, amint a diadalív felé haladnak. Az előtér alakjainak feje elválik a faltól – le is tört – a háttér figurái viszont laposan faragottak. A diadalív perspektivikusan ábrázolt.

\paragraph{Traianus diadaloszlopa}
Emlékműként álló oszlop, ami Traianusz császár dáciai hadjáratának állít emléket. Rajta spirálisan kb. 1 m magas, kiterítve 120 m hosszú domborműszalag meséli el a hadjáratot anélkül, hogy egyetlen kerettel megszakítaná az elbeszélés folyamatát. A jeleneteket csak a csoportszervezés különíti el egymástól némileg.

\begin{figure}[H]
	\begin{minipage}{0.5\textwidth}
		\tcbox[colback=darkgray!85!black,
		left=0mm,right=0mm,top=0mm,bottom=0mm,boxsep=1mm,toptitle=0.5mm,bottomtitle=0.5mm,
		title=\centering{Titus diadalívének domborműve}]{
			\includegraphics[width=1.0\linewidth]{03/titus_diadaliv}
		}
	\end{minipage}
	\hfill
	\begin{minipage}{0.45\textwidth}
		\tcbox[colback=darkgray!85!black,
		left=0mm,right=0mm,top=0mm,bottom=0mm,boxsep=1mm,toptitle=0.5mm,bottomtitle=0.5mm,
		title=\centering{Traianus diadaloszlopa}]{
			\includegraphics[width=1.0\linewidth]{03/traianus_diadaloszlopa}
		}	
	\end{minipage}	
\end{figure}

\subsubsection{Császárportrék}

\begin{wrapfigure}{r}{0.4\textwidth}
	\tcbox[colback=darkgray!85!black,
	left=0mm,right=0mm,top=0mm,bottom=0mm,boxsep=1mm,toptitle=0.5mm,bottomtitle=0.5mm,
	title=\centering{Császárportré}]{
		\includegraphics[width=1.0\linewidth]{03/csaszarportre}
	}
\end{wrapfigure}

	Alapvetően realizmus jellemzi a portrékat: a császárok valós vonásait tükrözik a szobrok.
	
	Néhány általános jellemző mégis mindegyiken megfigyelhető.
	
	Nem a fiatal, a görög és korábbi kultúráknál megszokott fiatal férfi jelenti az ideált, hanem az idősebb, tapasztal férfi válik eszményképpé.
	
	Erőteljes pofacsont, erősen kiugró szemöldökcsont, összeráncolt szemöldök, erős homlok és mosolyráncok, mélyen, sötét árnyékban ülő szemek, keskeny széles, összeszorított száj jellemzi a fejeket. Az arcok szigort, tekintélyt, bölcsességet, hatalmat tükröznek.

\subsubsection{Egész alakos figurák}

	\paragraph{Augusztusz egész alakos álló szobra}
	A szobor sok tekintetben követi a görög hagyományokat.
	A kontraposzt tartás és az izmos, részletesen kidogozott felsőtest a klasszikus kori szobrászat hatását mutatja (ilyen a Dárdavivő), a gazdagon, sűrűn redőző, dekoratív drapéria a hellenizmus hatására utal. (A római művészet kedveli és továbbviszi a görög hagyományokat.) A római szobrászat sajátosságai:
	Ezeket az arc kialakításában (szigorú, tekintélyt sugárzó arc) és a mellvéden található allegorikus figurákban találhatjuk meg.
	
	
	\paragraph{Marcus Aurélius lovasszobra}
	A műfaj: a bornz lovas szobor a római császárok kedvelt műfaja volt, ebben a császárok diadalmas hadvezérként lovon ülve ábrázoltatták magukat. Marcus Aurélius szobrán kívül azonban nem maradt fenn több példány. E szobor is annak köszönhette fennmaradását, hogy a középkorban azt hitték, hogy a szobor Nagy Konstantint, a kereszténységet államvallássá tévő császárt ábrázolja.
	
	\begin{figure}[H]
		\begin{minipage}{0.45\textwidth}
			\tcbox[colback=darkgray!85!black,
			left=0mm,right=0mm,top=0mm,bottom=0mm,boxsep=1mm,toptitle=0.5mm,bottomtitle=0.5mm,
			title=\centering{Augusztusz szobra}]{
				\includegraphics[width=1.0\linewidth]{03/augusztusz}
			}
		\end{minipage}
		\hfill
		\begin{minipage}{0.45\textwidth}
			\tcbox[colback=darkgray!85!black,
			left=0mm,right=0mm,top=0mm,bottom=0mm,boxsep=1mm,toptitle=0.5mm,bottomtitle=0.5mm,
			title=\centering{Marcus Aurelius lovasszobra}]{
				\includegraphics[width=1.0\linewidth]{03/marcus}
			}	
		\end{minipage}	
	\end{figure}

\subsection*{Az ókori Róma festészete}

A legtöbb festmény Pompeiiben maradt fenn, és a portré, a tájkép, a csendélet, a látszat-architektúra (építészeti tagozatok plasztikus, reális megfestése, imitálása) és a mitologikus témájú műfajokban készült.

\paragraph{Stílus} A realizmus, a fény-árnyékkal való modellálás, a plasztikusság jellemzi a képeket. A tájképeknél a reális térhatást a levegőperspektíva (a háttérben lévő motívumok egyre halványodnak, szürkés-kék színűekké válnak), a fény és a sötét, árnyékos felületek erős kontrasztja fokozza.

\paragraph{Technika} Falra kerülnek, leggyakrabban freskó, valamint mozaik.

	\subparagraph{Freskó technika} 
	Nedves vakolatra kerül a festék, és a vakolat mésztartalma a száradás során kiválva a falfelületen mészpáncélt hoz létre, így a freskó rendkívül tartós. Hátránya, hogy egy nap alatt kb. egy négyzetméternyi területet lehet megfesteni vele, amibe később már nem lehetett belenyúlni, nem lehet változtatni rajta.
	
	\subparagraph{Mozaik technika} A falra vagy a padlóra készül, apró kövek, festett kerámiák, színes üvegdarabokból kirakott kép.
	
\vspace{1cm}	

\tcbox[colback=darkgray!85!black,
left=0mm,right=0mm,top=0mm,bottom=0mm,boxsep=1mm,toptitle=0.5mm,bottomtitle=0.5mm,
title=\centering{Pompeii-i festmények}]{
	\includegraphics[width=0.34\linewidth]{03/romai_festeszet_1}
	\includegraphics[width=0.31\linewidth]{03/romai_festeszet_2}
	\includegraphics[width=0.33\linewidth]{03/romai_festeszet_3}
}

\clearpage

\subsection*{Pannónia provincia római kori maradványai}

\begin{wrapfigure}{r}{0.45\textwidth}
	\tcbox[colback=darkgray!85!black,
	left=0mm,right=0mm,top=0mm,bottom=0mm,boxsep=1mm,toptitle=0.5mm,bottomtitle=0.5mm,
	title=\centering{Pannonia}]{
		\includegraphics[width=1.0\linewidth]{03/pannonia}
	}
\end{wrapfigure}

Pannonia a Római Birodalom egyik provinciája volt. 

A rómaiak Pannonia provincia (és így a mai Magyarország területét) a Kr.u. 1. században hódították meg, több lépcsőben. A fejlődő kereskedelem miatt ekkor lett fontos ez a provincia. Kr.e.14-ig a Balatontól nyugatra lévő területeket foglalták el a római seregek (itt húzodott a Borostyánút), majd fokozatosan a Duna vonaláig haladtak. (Pannonia magában foglalja Kelet-Ausztriát, Szlovákia Dévény-Pozsony környéki részét, Észak-Szlovéniát, a Dunántúlt, Észak-Horvátországot, Észak-Szerbia egy részét és Bosznia-Hercegovina északi sávját is.) A mai Magyarország területén számtalan római település maradványait tárták fel.

\begin{wrapfigure}{r}{0.45\textwidth}
	\tcbox[colback=darkgray!85!black,
	left=0mm,right=0mm,top=0mm,bottom=0mm,boxsep=1mm,toptitle=0.5mm,bottomtitle=0.5mm,
	title=\centering{Aquincum}]{
		\includegraphics[width=1.0\linewidth]{03/aquincum}
	}
\end{wrapfigure}

\paragraph{Aquincum}
Budapest területén belül is több helyen találhatunk római emlékeket - a legismerebb, legjobban fennmaradt a mai Óbuda területén található Aquincum. Az egykori római települést és katonai tábort 89 körül alapították, virágkora a 2-3. századra esett, de egészen a 4. századik létezett. A főváros területén talált római emlékek legjelentősebb (és természetesen mozdítható) darabjait az Aquincumi Múzeumban tekinthetjük meg (itt találhatóak a római kori polgárváros maradványai is). 

\paragraph{Gorsium}
Gorsium hazánk egyik legjelentősebb római kori települése - a mai Tác község területén (Székesfehérvártól kb 15 km-re) található. A város már az 1. században létezett, több támadást is túlélt, végül hanyatlása a 4. században kezdődött meg, de az eredetileg 400 hektáros területén még évszázadokig laktak. A mai romterület, azaz a Gorsium Régészeti Park, nagyjából 200 hektáros, bejárása közel két órát vesz igénybe - a romok között palota, ókeresztény bazilika, közfürdő, hivatalos épületek, lakóházak és laktanya maradványai láthatóak. 

\paragraph{Brigetio}
Talán kévésbé ismert Brigetio neve - a mai Szőny (Komárom) területén a római időkben jelentős katonai tábor és polgárváros volt. Az ásatások során előkerült téglák, bronz pénzérmék, ékszerek és edények a komáromi Klapka György Múzeumban (a kőfaragványok az Igmándi erődben) és a tatai Kuny Domokos Múzeumban tekinthetőek meg. A dunai gátépítés közben azoban arra is fény derült, hogy a terület már a római időkben is fontos fürdőváros volt - hatalmas, mintegy 800 négyzetméteres fürdőkomplexum, valamint csatornázott, padlófűtéses házak romjait sikerült feltárni.

\begin{wrapfigure}{r}{0.45\textwidth}
	\tcbox[colback=darkgray!85!black,
	left=0mm,right=0mm,top=0mm,bottom=0mm,boxsep=1mm,toptitle=0.5mm,bottomtitle=0.5mm,
	title=\centering{Savaria}]{
		\includegraphics[width=1.0\linewidth]{03/savaria}
	}
\end{wrapfigure}

\paragraph{Savaria}
Savaria Szombathely római kori neve - ez a város volt a Borostyánút egyik állomása, a 2. századtól Felső-Pannonia vallási központja, később tartományi székhely. A római emlékek egy része a Járdányi Paulovics István Romkertben látható - itt az 1937-es ásatások során tártak fel jelentős épületcsoportokat: egy egykori palotát, és annak pazar mozaikpadlóját, római kori fürdőházat és Mercurius-szentély alapfalait is.


\clearpage

\section{A római kori festészet alapvető jellemzői}

\tcbox[left=0mm,right=0mm,top=0mm,bottom=0mm,boxsep=0mm,
toptitle=0.5mm,bottomtitle=0.5mm,title=\centering{A tétel adatai}]{%
\begin{tabular}{| p{0.25\textwidth} | p{0.7\textwidth} |}
	\hline
	Tétel teljes címe 
	&
	Egy konkrét alkotáson keresztül ismertesse a római kori festészet alapvető jellemzőit! Beszéljen a Pompeii-ben feltárt falfestmények alapján a négy meghatározó festészeti stílusáról! Mi az enkausztika?
	\\ \hline
	
\end{tabular}}

\subsection*{Pompeii festményeinek 4 stílusa}

A falfestés stílusai Pompejiben - a római festészet legszebb emlékeit Pompejiben és
Herculaneumban láthatjuk - a Vezúv Kr. u. 79-ben betemette a városokat, majd 1748-ban
fedezték fel - a falfestmények négy különböző stílust képviselnek, külön és keveredve is
előfordulnak.

\subsubsection{Összefoglalás}

\begin{enumerate}
	\item  \textbf{Inkrusztációs stílus} - a falfestmények berakásokat utánoznak, márványból készült
	falburkolat hatását keltik.
	
	\item  \textbf{Architektonikus stílus} - építészeti elemek ábrázolásával látszólag bővítették a teret. 
	
	\item \textbf{ Ornamentális stílus} - mitológia jeleneteket feldolgozó dekoratív stílus, keretezet
	táblakép illúzióját keltve.
	
	\item\textbf{ Illuzionisztikus stílus} - tájképek, csendéletek, épületek valószerűen ábrázolt látszatát
	festették - a mélység irányába futó vonalak párhuzamosak (axonometria) - használták a
	színperspektívát.
\end{enumerate} 

\tcbox[colback=darkgray!85!black,
left=0mm,right=0mm,top=0mm,bottom=0mm,boxsep=1mm,toptitle=0.5mm,bottomtitle=0.5mm,
title=\centering{Pompeii 4 stílusa}]{
	\includegraphics[width=1.0\linewidth]{03/"Pompeji 4 stilusa"}
}


\subsubsection{A falfestészet}

A pompeji ház minden szobáját kifestették, kivéve a gazdasági rendeltetésű helyiségeket
(éléskamra, konyha), valamint a rabszolgák lakószobáit. A falfestészet korántsem volt
állandó jellegű a város egész fennállásának folyamán: változott a megrendelők társadalmi
helyzete, változott az ízlés és ezzel együtt a festészet stílusa is változásokon ment
keresztül. A festők általában a freskó-technikát alkalmazták: a friss vakolatra festettek.
Az alapot előzőleg különleges eljárásokkal megdolgozták, hogy egy ideig friss maradjon,
s ne kezdjen száradni már a képfestés közben. A pompeji falfestmények frissen csillogó
színei részben az alap előzetes preparálásának tulajdoníthatók, részben azonban annak is,
hogy a festékek kitűnő minőségűek; a jó festékkeverés nemzedékek kitartó
fáradozásainak és kísérletezéseinek volt az eredménye.


Pompejiben több szépmíves műhely vagy „iskola” működött, melyek mindegyike
kialakította a maga külön művészi szabályait s tagjaik szigorúan ragaszkodtak ezekhez. A
művészi önállóság csupán abban nyilvánulhatott meg, hogy a művész a saját
egyéniségének megfelelően kombinálta és fejlesztette a hagyományos motívumokat. 
Pompeji persze nem volt hangadó centrum; a Rómában uralkodó áramlatokat követte,
Róma pedig a Görögországból és a hellénisztikus Keletről származó irányzatokat. A
pompeji falfestészetben négy stílust különböztetünk meg.

\subsubsection{Inkrusztációs stílus}

Az első stílus az „inkrusztációs stílus” nevet kapta. Lényege abban áll, hogy a falak
vakolatát márványszerűen festették ki, s ily módon a falak azt a benyomást keltik a
szemlélőben, mintha márványlapokból lennének összeillesztve. A falfelületet először
több mezőre osztották, melyeket különféle módokon lehetett kialakítani; majd az egyes
mezőket tarka színekkel bemázolták, s márvány-ereket utánzó vonalakkal hálózták be.
Ennek a stílusnak jó példáját láthatjuk Sallustius házának tablinumában.


A faltő végig sárga, fölötte fekete márványtáblákat utánzó széles sáv húzódik, majd több
sorban elhelyezett keskeny, különböző színű (sárga, zöld, lila) téglalapok következtek. A
festőn kívül még stukkókészítő mester is közreműködött a ház díszítésében. A fal felső
részén stukkóból formált párkányfogazat vonul végig, fölötte ismét színes „márványtáblák” sorai helyezkednek el, majd újra párkányzat következik. Ily módon a
falnak mintegy kétharmadát díszítések borítják.
Ennek a stílusnak teljes-kibontakozását főleg Délos szigetén és a kisázsiai Priéné
városban épült hellénisztikus házak falain figyelhetjük meg. Itáliai változata - melyet
éppen Pompejiben ismerhetünk meg legjobban - kiváltképpen a tisztán architektonikus
részleteket kedvelte: a fal függőleges voltát kihangsúlyozó oszlopokat és pilléreket.
Ezeket elsősorban ajtók, ablakok, fülkék keretezésére alkalmazták, valamint a fal felső
részén, az első és a második párkányzat között. Általában véve ez a stílus kissé merev.
Súlyossága idővel fárasztónak hat, s a fal unalmasnak tűnik.

\subsubsection{Architektonikus stílus}

A második stílus (i. e. II. század vége) elhagyta a stukkó-párkányzatokat, és az első
stílusnak csupán a festészeti elemeit alkalmazta. Tisztára architektonikus festészetnek
lehet nevezni. Az első stílus „megszilárdította” a falat, korláttá varázsolta, mely a
helyiséget elválasztotta a külvilágtól; a második stílus azonban eltávolította ezt a korlátot.

Az oszlopok, féloszlopok és pillérek most már nem csupán az ajtókat keretezték be, s
nemcsak a fal felső részének díszítésére szolgáltak; a művész az egész falfelületet
befestette velük a faltőtől - sőt néha a padlótól - egészen a mennyezetig.
Az arányok gondos mérlegelése és az árnyékok művészi elosztása folytán az igazi fal
eltűnt, s mint a színpadi dekorációknál, elrejtőzött a művész alkotta látszatvalóság
mögött. A festő a perspektíva eszközeivel a térbeli mélység illúzióját varázsolja a néző
szeme elé. A második stílus további fejlődése folyamán a falon monumentális épületek
jelennek meg térbeli egymásmögöttiségben elrendezve: a szemlélőben elenyészik a fal
határoló, korlátozó jellegének érzete. A fal közepére - az épületek közé - rendszerint zárt ajtót festett a művész. A második stílus utolsó periódusában ez az ajtó „megnyílt”, s a szemlélő rajta „kitekintve” fákat, embereket, állatokat - tájakat és mitológiai jeleneteket
láthatott.

A második stílus két különböző elemet egyesített: az architektonikus perspektívát és a
monumentális falfestészetet, melynek az előbbi keretül szolgált. E stílus sikerének titka
részint a faldekorációk gondolati egyszerűségében és tisztaságában rejlik, részint pedig
abban, hogy a művészek pompás architektonikus képeiknek meg tudták adni azt a 
perspektivikus mélységet, mely gyakran csalódásig híven utánozza a valóságot. Ezek az
architektúrák nemcsak a falakat „szüntették meg”, hanem magát a helyiséget is - olyan
kilátóponttá varázsolták, ahonnan gyönyörködni lehetett büszke paloták szépségében és a
mögöttük elterülő táj végtelenbe nyúló messzeségében. Ennek a stílusnak Kis-ázsia
hellénisztikus központjaiban volt a hazája; bajosan lehetett volna még egy ilyen festészeti
stílust találni, amely ennyire illett volna a római uralom terjeszkedési korához.

\subsubsection{Ornamentális stílus}

A harmadik stílus bizonyos értelemben tiltakozást fejezett ki a második ellen. Az utóbbi
stílus - mint mondottuk - „eltűnteti” a falakat: a térbeli mélység illúzióját keltő
épületekkel festi tele őket, melyek mögött még tájak és emberek láthatók. Ezzel szemben
a harmadik stílus sohasem téveszti szem elől, hogy a fal mindenek előtt sík felület: tehát
nem „töri át” sehol, s nem törekszik térbeli, perspektivikus hatásokra. A harmadik stílus
tökéletes példáját láthatjuk a „Százéves jubileum házában”, ahol a fő mező fekete
hátterén mintegy elenyészve apró fehér figurák állnak. Ennek a stílusnak a dekorációi
faliszőnyeget utánoznak.
Pompejibe természetesen voltak igazi faliszőnyegek is, de egy sem maradt meg; csak a
falba vert szögek és a kampók jelzik egykori helyüket. A művész egész képsorozatokat
ábrázolt faliszőnyeg-utánzatán. De hiába keresnénk köztük a római történelemből vagy a
helyi krónikából vett jeleneteket viszont majdnem az egész görög mitológiát láthatjuk
rajtuk, főleg szerelmi történeteket, melyek Ovidius mitológiai költészetének szellemét
sugározzák. 

\subsubsection{ Illuzionisztikus stílus}

A negyedik stílus korai formájának kell tekintenünk azt a kísérletet, amely a „szőnyegelv” és a perspektíva összeegyeztetésére törekedett. Ez persze megvalósíthatatlan, mert a
perspektíva „kitárja”, a „szőnyeg” ellenben „bezárja” a falat. Ennek a próbálkozásnak
egyik példáját láthatjuk Lucretius Fronto házában. A faltő helyett csupán keskeny sáv fut
végig a padlózat fölött; a fal öt mezőre van tagolva; a középső mező sötétpiros „szőnyeg”, rajta Dionysost és Ariadnét ábrázoló kép; ettől jobbra és balra egy-egy fekete „szőnyeg”, arany kandeláberekkel és kis tájképekkel díszítve.
A negyedik és az ötödik sávban architektonikus motívumok szerepelnek: alul nagy ablak
oromtetővel; fölötte - a perspektivikus ábrázolás halvány visszfényeként - egy kerek
épület ión oszlopai, melyek túlságosan vékonyak ahhoz, hogy valóságérzetet
kelthessenek; a körbefutó virágfüzérek még csak jobban kihangsúlyozzák az
architektonikus motívum valószerűtlenségét. Éppen ilyen fantasztikusak a felső fríz
épületei, bár közöttük valóságosan létező típusokat is lehet látni: a középen egy
háromajtós bazilikát, s mindegyik oldalon egy-egy ctediculát, felettük fedélszerkezettel.
Ám a bazilika közepét kis csendélet foglalja el, fölötte pedig háromlábú üst (tripus)
látható - mintha a művész figyelmeztetni akarná a nézőt, hogy kompozícióját
semmiképpen ne tekintse egy valóságos épület ábrázolásának.

A harmadik stílus - melynek finomságát Pompejiben meg sem közelíti a többi - túlságosan „akadémikus” volt ahhoz, hogy nagy sikerre számíthatott volna. S ha a
köztársaság korának megrázkódtatásaitól, a császárság korának üldöztetéseitől kimerült 
régi arisztokrácia a házaiba vonult vissza, hogy legalább otthon élvezzen ideig-óráig
nyugalmat, biztonságot és békességet - az Imperium támogatását élvező kereskedelem és
ipar képviselői még otthonukban sem óhajtottak elszakadni a világtól; házi kényelmet
akartak, de nem kívántak lemondani a világ csábító messzeségeiről.
A „szőnyeg-stílus” és a perspektíva összehangolására irányuló kísérlet ebből a
szemléletből, ebből a lelki beállítottságból fakadt. A város pusztulása természetesen véget
vetett a pompeji falfestészet további fejlődésének. Itt-ott meg lehet figyelni újabb stílusok
csíráit, de ezek már nem bontakozhattak ki.


\subsection*{A misztériumok villája}

\tcbox[colback=darkgray!85!black,
left=0mm,right=0mm,top=0mm,bottom=0mm,boxsep=1mm,toptitle=0.5mm,bottomtitle=0.5mm,
title=\centering{A misztériumok villájának freskója}]{
	\includegraphics[width=1.0\linewidth]{03/miszteriumok_villaja}
}

A Misztériumok villája Kr. e. 2. században épült, majd az 1. század során újjáépítették. Egyike a több mint 100 római villának, amelyeket a Vezúv környékén feltártak. A Herculaneumi-kaputól nem messze található a Via dei Sepolcri mentén.

Lakóhelyiségeit a második pompeii korszak (architektonikus stílus) stílusjegyeit viselő falfestmények díszítik. Az egyik legérdekesebb freskó egy misztériumjátékot (rituálét) ábrázol, amely során egy fiatal nőt beavatnak a házasság misztériumába (innen származik az épület elnevezése is). Ugyanakkor számos harmadik pompeii korszakbeli falfestmény is található, ezek főleg egyiptomi motívumokat ábrázolnak.

\paragraph{Architektonikus stílus jellemzői}
\begin{compactitem}
	\item Párkányzatok, oszlopok, építészeti elemek tagolják, keretezik a képeket.
	
	\item A képek a tér folytatásaként jelennek meg, kitágítják a teret. Olyan hatást keltenek, mintha a jelenet a szoba folytatásában, a szemlélő körül játszódnának.
\end{compactitem}

\subsection*{Enkausztika}

\subsubsection{Története}
Az enkausztika az egyik legrégebbi festőtechnika. Egyben a legidőtállóbb is.
Magát a méhviaszt, mint festék-kötőanyagot i.e. 3000 óta ismerik.
Egyiptomból az i.e. 2. századból maradtak fenn fatáblára enkausztikával festett portrék, frissességükből mit sem veszített, ép állapotban. Ezek a legrégibb fenn maradt emlékei e technikának, melynek eredete még régebbre nyúlik vissza Görögországban. Kezdetben csak ipari célokra alkalmazták a technikát, hajók festésére.

Az enkausztika lényege az, hogy \textbf{melegen folyós viaszfestékkel festenek}.
Mivel csak melegen alakítható a viaszfesték, nehéz vele dolgozni. Maga az enkausztika szó is melegen való festésre utal.

A viasz a mézgyártás mellékterméke és az ókortól fogva mindig nagy mennyiségben állt rendelkezésre. Plinius történeti írásai szerint a görögök igen régi időktől fogva alkalmazták ezt a technikát. Írott források falfestészetben is megemlítik alkalmazását. Fennmaradt néhány viaszfestő művész neve is, mint pl. Pausias, aki virágcsokrokat és emberalakokat festett magas színvonalon, amiért nagy hírnévre tett szert a maga korában.

Az enkausztikával készült festményeket gyönyörű mély fény, egymásba lágyan olvadó, tüzes színek jellemzik.

Valójában még a mai napig sem sikerült kideríteni, hogy az ókori görögök hogyan készítették el a viaszfestéket és miként dolgoztak vele. A fennmaradt írásos adatok arról számolnak be, hogy ún. punviaszt használtak hozzá. Ezt tengervízben szóda hozzáadásával felforralták, majd a holdfény hatásának tették ki.

Bonyolult eljárás során készültek el a festék pigmenteket tartalmazó színes viaszpaszták, melyeket alapozatlan fatáblára vittek fel egy hosszú nyelű fémeszközzel és meleg állapotában simították el. Kis kályhákkal melegítették mind a fémeszközöket, mind a viaszt. Ilyen fémeszközöket régészeti feltárások során is találtak.

Később, az ókort követően, feledésbe merült e tartós technika és csak a 19. századtól kezdték újra felfedezni.

\subsubsection{A technika jellemzői}

\paragraph{Kötőanyag}
Az enkausztika technikánál a méhviasz a kötőanyag. Kizárólag melegítve használható festésre, amikor folyóssá válik.

\paragraph{Hordozó}
Az enkausztika festményalapjai lehetnek: fatáblák, farostlemez, fémlap, papír, vászon. Egykor kőre, márványra, agyagtáblára is készültek festmények.

\paragraph{Alapozás}
Az ókorban alapozott és alapozatlan felületekre festettek ezzel a technikával. Manapság sem szükséges alapozás a farostlemezekre, de a rajz - és akvarell papírokra sem. Ha mégis alapozzák, viasszappant keverenek az alapozószerbe és vékonyan kenik fel. Vásznakra is így készül alapozás.

\paragraph{Pigment}
Minden fényálló pigment alkalmas ehhez a technikához, finoman őrölve. Csak azokról a pigmentekről kell lemondani, amelyek nem viselik el a 80 C fok fölötti hőmérsékletet (a viasz olvadása miatt). Házilag való elkészítésük nagy tapasztalatot igényel.

\paragraph{A technika alapanyagjának nehézségei}
A 20. században a legelső enkausztika festéket gyártó gyár csődbe ment, a kereslet csekély volta miatt. Igen kevés művész alkalmazza ma is ezt a technikát és kevés cég gyárt hozzá festékeket, sokszor nem is az eredeti technikának megfelelően. Ilyenek pl. a viaszfestékkréták. Bár házilag nagyon nehéz előállítani az enkausztika festéket, azok a művészek, akik használják, mégis maguk készítik el.

Ez a már megszűnt, egykori cég kockák formájában árulta a viszfestékeket és elektromosan fűthető palettákat kínált hozzájuk. Ecsettel festettek, és ha félretették az ecsetet, rögtön megdermedt benne a viaszfesték, amikor kihűlt, de felmelegítve újra folyóssá vált. A festőfelületet is melegítették. Az ecseteket lakkbenzinnel mosták ki használat után.

Maga a viasz víztaszító és igen ellenálló tulajdonságú. Gyártása során kifehérítik. Olvadáspontja 63-65 C fok körüli.

Festék készítésekor a pigmenteket összeolvasztják a méhviasszal. A festék eloszlatása a festőfelületen is hőközléssel történik.

\paragraph{Eszközök}
Elektromosan fűthető palettákat használnak a viaszfesték olvadt, folyékony állapotban tartásához. Írópulthoz hasonló állványra kerülnek a hordozók. Hősugárzókkal történik a festőfelület melegítése. Festéshez a sörteecsetek és a spatulák a legmegfelelőbbek.

A fűtött spatulák manapság gyengeárammal működnek, így teljesen veszélytelenek. A spatulák különböző formájú és méretű spatulapapucsokkal egészíthetők ki. A kézi hősugárzókat manapság infravörös lámpákkal is helyettesítik.

A festőfelületet elölről vagy hátulról melegítik. A szilárd, kemény enkausztika festékek kb. 70 fokon válnak folyékonnyá és festhető állagúvá.

A mai enkausztik technika az elektromosan fűthető eszközök segítségével sokkal könnyebbé vált, mint az ókorban.

\paragraph{Folyamat}
A festmények vázlatai készülhetnek szénnel, krétával, ceruzával, akvarellel stb. a hordozókra. Erre kerülnek azután a viaszfestékek.

Az elkészült képet, ha megdermedt a kihűlt viasz, gyapjú ronggyal utólag csillogóvá polírozhatjuk.

A technika előnyei: a viaszfestmény nem sárgul, nem repedezik, nem zsugorodik, ellenáll a nedvességnek, időtálló évezredeken keresztül, ahogy a fennmaradt képek is bizonyítják. Gyakorlatilag örökéletű. Egyetlen veszély fenyegeti: a hőhatás, amitől megolvadhat vagy eléghet. Sajnos a történelem viharaiban, háborúk és tűzvészek során ennek estek áldozatul az enkausztika festmények.

A festést, ha abbahagyjuk, megdermed a festék, amit akár évek múlva is folytathatunk, újra felmelegítve.
\cleardoublepage

\chapter{Az ókeresztény és a bizánci művészet} % Introduction
\label{ch:4_okereszteny_bizanc}

\section{Az ókeresztény és a bizánci építészet, festészet és díszítőművészet}

\begin{center}
	\begin{longtable}{ | p{0.25\textwidth} | p{0.75\textwidth} | }
		
		\hline
		\multicolumn{2}{|c|}{\textbf{A tétel adatai}}
		\\ \hline
		
		\hline
		Tétel teljes címe
		&
		Mutassa be az ókeresztény és a bizánci művészetet - az építészet és a festészet, valamint a díszítőművészet stílusjegyeit, ismert alkotásait!
		\\ \hline
		
		Jegyzetek
		&
		\begin{compactitem}
			\item Az ókeresztény és bizánci építészet - a bazilika és a bizánci templom felépítése.
			\item Az ókeresztény szobrászat stílusjegyei és jelképei.
			\item A bizánci mozaik, az ikonfestészet és a kézművesség jellemzői.
		\end{compactitem}
		\\\hline
		
	\end{longtable}
\end{center}

\cleardoublepage


\section{Az ikonfestészet}

\begin{center}
	\begin{longtable}{ | p{0.25\textwidth} | p{0.75\textwidth} | }
		
		\hline
		\multicolumn{2}{|c|}{\textbf{A tétel adatai}}
		\\ \hline
		
		\hline
		Tétel teljes címe 
		&
		Milyen szabályrendszer határozza meg az ikonok képi világát? Mutassa be az ikonfestés technikáját, lépéseit! Milyen anyagok szükségesek egy ikon festéséhez? Milyen eszközök szükségesek az aranyozáshoz?
		\\ \hline
		
	\end{longtable}
\end{center}

\cleardoublepage

\chapter{A románkori művészet} % Introduction
\label{ch:5_romankor}

\section{A románkori építészet, szobrászat és kézművesség}


\tcbox[left=0mm,right=0mm,top=0mm,bottom=0mm,boxsep=0mm,
toptitle=0.5mm,bottomtitle=0.5mm,title=\centering{A tétel adatai}]{%
	
	\begin{tabular}{| p{0.25\textwidth} | p{0.7\textwidth} |}

		\hline
		\centering{Tétel teljes címe}
		&
		Mutassa be a románkori építészet, szobrászat és kézművesség stílusjegyeit! Jellemezze a különböző műfajokhoz kapcsolódó alkotásokat, európai és hazai műemlékeinket!
		\\ \hline
		
		\centering{Jegyzetek}
		&
		\begin{compactitem}
			\item A románkor társadalma, hitvilága, építészete.
			\item A koraközépkori szobrászat, kódexfestészet és kézművesség jellemzői.
		\end{compactitem}
		\\\hline
		
	\end{tabular}}

	\subsection*{Bevezetés}
	
		\paragraph{Földrajzi elhelyezése}
		A román kor a népvándorlás kora után kialakuló nyugat-európai (elsősorban a német, francia nyelvterületek és itáliai), kereszténységet fölvett államok művészete.
		Az első keresztény állam létrejöttétől (Frank Birodalom Nagy Károly vezetésével kb. 800-tól) a IX., X., XI. századon át a XII. század végéig, a gótika elterjedéséig számítjuk.
		
		\paragraph{Korszakai}
		A IX., X. századot korai román, vagy preromán („pre” = valami előtti) kornak nevezzük.
		A XI., XII. század a korszak virágkora, ezért ezt érett román kornak nevezzük.
		
		\paragraph{Kifejezés eredete, jelentése}
		A kifejezés 19. századi eredetű. A nyugati nyelvekből került magyar nyelvbe, a nyugati szakkifejezés hangalakjának átvételével. A magyar kifejezésnek azonban semmi köze a román nyelvhez és népcsoporthoz, annál több az antik római kor építészetéhez. A román kori építészetben az ókori Róma építészetének hatását, vonásait fedezhetjük fel – a kifejezés jelentése tehát nem „román”, hanem római.
	
		\paragraph{Római hatások}
		\begin{compactitem}
			\item A román kor építészete az antik római építészet hatását mutatja: azzal kapcsolódik a római építészethez leginkább, hogy nyílásai, boltozatai félkörívesek. (A 19. században „félköríves stílnek” is nevezték megkülönböztetve az utána következő gótikától, ami a „csúcsíves stíl” nevet viselte).
			
			\item A román kor művészete abban „római”, „rómaias” még, hogy a vele egy időben virágkorát élő Bizánc művészetével szemben nem a keleti, ortodox, hanem a nyugati, Róma-központú keresztény egyház művészete.
			
			\item További „római” vonás a korban, hogy a korai román kor uralkodói – Nagy Károly, Ottók – arra törekedtek, hogy államaikkal az egykori antik nyugat-római birodalmat élesszék fel egyfajta „keresztény római birodalom” formájában.
		\end{compactitem}
	
	\subsection*{Kultúrtörténeti háttér}
		
		A román kor az Európába vándorolt „barbár” népek letelepülésével, államalapításával, a kereszténység felvételével kezdődik. A legjelentősebb ilyen megszilárdult keresztény állam a Frank Birodalom. A Frank Birodalom uralkodója Nagy károly, aki a római pápával 800-ban császárrá koronáztatja magát. Császársága az egykori nyugat-római birodalom jogutódja kíván lenni, és a Keleten virágzó Bizánc nyugati riválisa is, ezért fél évszázadnyi időre Nagy Károly Európa nagy részét fennhatósága alá vonja.
		
		\paragraph{A Német-Római Birodalom kialakulása}
		a X. században a német területek erősödnek meg: a német Ottó nevű uralkodók veszik át az irányítást. A frank állam hanyatlása, majd a kalandozó magyarok legyőzése után I. Ottó koronáztatja magát császárrá Rómában, és létrehozza a Német-Római Birodalmat szintén az egykori antik Róma felélesztésének szándékával. A Nyugat-Római Birodalom a XI-XII. században is Európa legnagyobb hatalommal bíró állama marad.
		
		\paragraph{A pápai állam vagy egyházi állam}
		A román kor alatt még egy jelentős állam jön létre: az Egyházi állam Közép-Itália területén. Ezzel a nyugati egyház politikai hatalomra is szert tesz Európában, és a művészetek legfőbb megrendelőjévé válik: templomokat épített, azok szobrászi és festészeti díszítését irányítja. Így a román kor művészete nagy részt szakrális, azaz vallásos célú. Tovább fokozza az egyház hatalmát Nyugat- és Közép-Európa területén az ún. egyházszakadás: 1054-ben az addig egységes keresztény egyház ketté válik: nyugati-római egyházzá és keleti, ortodox egyházzá.
		
		\paragraph{Szerzetesrendek}
		A román korban igen nagy szerepet töltenek be a népvándorlás alatt megszülető szerzetesrendek, melyek a betelepülő „barbár”népek megtérítésének, majd az új keresztény államokban a keresztény vallás terjesztésének érdekében jönnek létre. A terjesztésben a művészetet hívják segítségül: a templom-építészetben, kolostor-építészetben, a templomok, kolostorok szobrászati díszítésében, a kódexfestészetben jeleskednek. A leghíresebb román kori szerzetesrend a bencés rend (Szent Benedek alapítja).
		
		\paragraph{Keresztes hadjáratok}
		A román kor idején az egyház keresztes hadjáratokat indít, melyek a pápa által szentesített, a keresztes lovagok részvételével folytatott nagyarányú hadjáratok voltak a 11–13. században. Fő céljuk a Szentföld megszerzése volt a muszlim araboktól és törököktől, de a spanyol nyelvterületen előrenyomuló mórok (spanyolországi arabok, afrikaiak) ellen is.
		
		
	\subsection*{Templomépítészet}
	
	Ebben a koprszakban templomok, kolostorok és várak épülnek leginkább.
	
	\subsubsection{Külső jellemzők}
	
	A templom az egyház megerősödő hatalmát, az egyház által nyújtott védelmet van hívatva kifejezni, ezért külseje biztonságot, erőt, megingathatatlanságot sugall.
	\begin{compactitem}
		\item A külső zárt, csak kisméretű nyílásokkal van áttörve.
		\item A tömegek zárt, egyszerű, tömbszerű, gyakran egészen a geometriai.
		formákig leegyszerűsített hasáb-, henger- és kockaszerű egységek.
		\item A templomot vaskos, masszív, robosztus tömegek alkotják.
		\item A külső puritán, azaz alig díszített.
	\end{compactitem}

	\subsubsection{Alaprajz}
	
	A templom alaprajza a keresztény egyház szertartásainak rendje szerint alakul: követi az ókeresztény bazilikák formáját.
	\begin{compactitem}
		\item Hosszanti elrendezésű,
		\item három vagy öt hajós,
		\item keletelt: azaz keleti oldalán helyezkedik el a félkörívű apszis, a szentély,
		\item az apszis előtt négyezeti tér és kereszthajó található.
	\end{compactitem}

	\subsubsection{Szerkezet}
	
	\begin{wrapfigure}{r}{0.2\textwidth}
		\tcbox[colback=darkgray!85!black,
		left=0mm,right=0mm,top=0mm,bottom=0mm,boxsep=1mm,toptitle=0.5mm,bottomtitle=0.5mm,
		title=\centering{Kockafejezet}]{
			\includegraphics[width=1.0\linewidth]{05/kockafejezet}
		}
	\end{wrapfigure}
	
	A templomok bazilikális szerkezetűek, azaz
	főhajójuk kiemelkedik a mellékhajók fölött, és a belső
	tér a kiemelkedő falakon nyitott ablakokon keresztül
	kapja a megvilágítást.
	
	Az ablakok kis méretűek, a falak tehát kevéssé vannak áttörve, vastagok,
	masszívak, erősek, további külső megtámasztás nélkül viselik a boltozat súlyát.
	
	A román templomok belső súlyt hordó tartóelemei
	egyszerű, hasábszerű pillérek, vagy tömzsi oszlopok. Az oszlopfejezetek gyakran minden díszítést nélkülöző, kocka formájú oszlopfők – ezek az ún. „\textbf{kockafejezetek}”.
	
	\subsubsection{Külső díszítőelemek}
	
	A templomok kevéssé díszítettek, de a falfelületeket nyílások, függőleges és vízszintes elemek tagolják.
	
	\paragraph{Bélletes kapuzat}
	A román kori templomok legjellemzőbb homlokzati
	eleme, a befelé szűkülő, a kapu körül ívben végigfutó féloszlopokkal díszített
	félköríves kapuzat. A bélletek (a kapu fölött futó féloszlopok) gyakran
	geomertikus mintákkal, fogsor-motívummal, cikk-cakk mintával faragott.
	
	\paragraph{Lőrészszerű ablak}
	Kis ablaktípus, keskeny, felül íves, befelé szűkül.
	
	\paragraph{Ikerablak}
	Kettős vagy hármas ablak, felül ívesek (félkör) az ablakok
	között kis oszlopok vannak.
	
	\paragraph{Ívsor vagy ívpárkányzat}
	Vízszintes tagolóelem, ami a külsőn jelzi a belső
	szinteket: kis ívek sorakoznak egymás mellett.
	
	\paragraph{Törpegaléria}
	Apró árkádok (kis boltívek és kis oszlopok), árkád a homlokzaton, ami mögött egy keskeny fal húzódik.
	
	\paragraph{Vakárkád}
	A homlokzaton ált. alul található olyan árkádok ami alatt, nem lehet
	átmenni, mert be vannak falazva tehát csak díszítő szerepük van.
	
	\paragraph{Féloszlop}
	A falhoz tapadó, függőlegesen félbevágott oszlop (oszlop-lábazata,
	oszlopfője van).
	
	\paragraph{Pilaszter}
	A falból négyzetesen kiugró (oszlopszerű) falsáv, amelynek fejezete és
	lábazata van.
	
	\paragraph{Lizéna}
	A falból négyzetesen kiugró falsáv fejezet és lábazat nélkül.
	
	
	\subsubsection{Térlefedés}
	
	A templomok boltozási rendszere az antik Róma idején kialakult boltozást: a donga és keresztboltozatot fejleszti tovább.
	
	A leggyakoribb boltozattípus a „román
	keresztboltozat” = két félhenger alakú dongaboltozat
	kereszteződése négyzet alaprajzon.
	
	A román keresztboltozatnak két típusa alakult ki.
	
	\paragraph{Élkeresztboltozat}
	Az íves falrészek megerősítés nélkül élekben találkoznak.
	
	\paragraph{Bordás keresztboltozat}
	az íves falrészek találkozásánál a boltozatot bordákkal erősítik meg, a bordák, amiket erős, súlyosabb kövekből raknak ki levezetik a boltozat súlyát az oldalfalakra, így a köztük lévő falakat könnyebb kövekből lehet rakni (Ez a típus alakul ki időben később, a XII. sz.-ban).
	
	\paragraph{Kötött térrendszer}
	A boltozás rendszere, logikája: ún. „kötött térrendszer” v. ,,kötött boltozás”. Mivel csak négyzet alaprajzú tereket tudtak lefedni, kialakult az a rendszer, hogy a főhajó egy boltszakaszára a mellékhajóban két fele akkora négyzetes boltszakasz kerül: hevederívek, a boltszakaszokat egymástól elválasztó ívek.
	
	\begin{figure}[H]
		\centering
		\begin{minipage}{0.45\textwidth}
			\tcbox[colback=darkgray!85!black,
			left=0mm,right=0mm,top=0mm,bottom=0mm,boxsep=1mm,toptitle=0.5mm,bottomtitle=0.5mm,
			title=\centering{Keresztboltozat}]{
				\includegraphics[width=1.0\linewidth]{05/keresztboltozat}
			}
		\end{minipage}
		\hfill
		\begin{minipage}{0.5\textwidth}
			
			\tcbox[colback=darkgray!85!black,
			left=0mm,right=0mm,top=0mm,bottom=0mm,boxsep=1mm,toptitle=0.5mm,bottomtitle=0.5mm,
			title=\centering{Kötött térrendszer}]{
				\includegraphics[width=1.0\linewidth]{05/kotott_terrendszer}
			}
		\end{minipage}
	\end{figure}

	\subsubsection{Német építészet}
	
	A német román templomok leghíresebb példái:
	a Német- Római császárok által épített dómok, ún. ,, császárdómok”: Hildesheim-i Szent Mihály templom, Maria Laach dómja, Speyer-i császárdóm, Worms-i császári dóm.
	
	\begin{figure}[H]
		\centering
		\begin{minipage}{0.3\textwidth}
			\tcbox[colback=darkgray!85!black,
			left=0mm,right=0mm,top=0mm,bottom=0mm,boxsep=1mm,toptitle=0.5mm,bottomtitle=0.5mm,
			title=\centering{Hildesheim-i Szent Mihály templom}]{
				\includegraphics[width=1.0\linewidth]{05/hildesheim.jpg}
			}
		\end{minipage}
		\hfill
		\begin{minipage}{0.3\textwidth}
			
			\tcbox[colback=darkgray!85!black,
			left=0mm,right=0mm,top=0mm,bottom=0mm,boxsep=1mm,toptitle=0.5mm,bottomtitle=0.5mm,
			title=\centering{Maria Laach dóm}]{
				\includegraphics[width=1.0\linewidth]{05/maria_laach}
			}
		\end{minipage}
	\hfill
	\begin{minipage}{0.34\textwidth}
	
		\tcbox[colback=darkgray!85!black,
		left=0mm,right=0mm,top=0mm,bottom=0mm,boxsep=1mm,toptitle=0.5mm,bottomtitle=0.5mm,
		title=\centering{Speyer-i császárdóm}]{
			\includegraphics[width=1.0\linewidth]{05/speyer}
		}
	\end{minipage}
	\end{figure}
	
	
	\subsubsection{Itáliai építészet}
	
	\paragraph{Sajátosságai}
	Az itáliai építészet sajátosságai, hogy
	\begin{compactitem}
		\item a templomok három épületegységből állnak
		\item gyakori a márványburkolat
		\item gyakori az ún. „oroszlános kapuzat”
	\end{compactitem}

	\paragraph{Épületei}
	A templomhoz, azaz a dómhoz további két, külön álló épület kapcsolódik. A campanile, azaz harangtorony és a baptisztérium, azaz keresztelő kápolna. A kifejezés a latin baptistero (jelentése keresztelés) szóból ered. A
	baptisztérium olyan épület, ami a templom nyugati
	homlokzatával szemben áll, centrális alaprajzú, és benne
	egy szintén centrális keresztelő medence található a
	keresztelés szertartásának végzésére.
	
	A három részes román épületegyüttesek leghíresebb példája:
	a Pisa-i épületegyüttes.
	
	\paragraph{Pisai dóm}
	
	\subparagraph{Alaprajz}
	Alaprajza olyan mintha 3 bazilikából állna: 5-hajós hosszház + 2 háromhajós
	keresztház.
	
	\subparagraph{Külső tömeg}
	Hangsúlyos négyzeti tér alakul ki, ami fölött az itáliai
	építészetben a német templomok szögletes, sátortetős négyzeti tornyaitól
	eltérően íves, kupolához hasonló térlefedés van.
	
	\subparagraph{Homokzat}
	Legfőbb homlokzati díszek a törpegalériák (4 szinten), alul
	vakárkádok. A homlokzat teljes felületét márványburkolat díszíti
	
	\subparagraph{Belső tér} Márvány burkolat (csíkos), nyitott fa fedélszék
	
	\subparagraph{Szerkezet} Jól láthatóan bazilikás.
	
	\subparagraph{Pisai keresztelőkápolna (baptistérium)}
	Alaprajza kör alakú.
	
	Külseje követi a dóm díszítését - vakárkádos, törpegalériás ez is
	(a felső szintek már a gótika jegyében készültek, ezért felül fiatornyok, növényi faragványok is láthatók).
	
	
	
	\begin{wrapfigure}{r}{0.45\textwidth}
		\tcbox[colback=darkgray!85!black,
		left=0mm,right=0mm,top=0mm,bottom=0mm,boxsep=1mm,toptitle=0.5mm,bottomtitle=0.5mm,
		title=\centering{Pisa-i épületegyüttes}]{
			\includegraphics[width=1.0\linewidth]{05/pisa}
		}
	\end{wrapfigure}
	
	\subparagraph{Pisai "ferdetorony" (campanile)}
	Illeszkedik a dóm és a baptisztérium külső díszítéséhez: alul itt is vakárkádok, a felső szinteken törpegalériák találhatók; legfelül félköríves nyílások között a harangok; kör alaprajzú
		
	Már építése közben is elkezdett süllyedni, a 3 szint után megváltoztatták a dőlésszöget, hogy ne ferdüljön tovább, de már késő volt, az épület tovább süllyedt.
	
	\paragraph{Márványinkrusztáció}
	Az itáliai templomok különösen Firenzében elterjedt, gyakori díszítési módja a márványinkrusztáció, azaz színes márványlapokból kialakított burkolat. A márványlapok gyakran geometrikus mintát mutatnak, vagy faltagoló elemeket, pl. törpegalériát, ablakot, kapunyílást imitálnak.
	
	\paragraph{Oroszlános kapuzat}
	Észak-Itália templomainak leggyakoribb jellemzője az „oroszlános kapuzat”
	Az egyébként viszonylag díszítetlen templomhomlokzaton a kapu fölött baldachin található, aminek súlyát egy-egy oroszlánra támaszkodó oszlop tartja. A kaput őrző oroszlán motívuma a művészettörténetben a védelem szimbóluma.
	
	\subsubsection{A Cluny bencés apátság}
	
	\begin{wrapfigure}{r}{0.45\textwidth}
		\tcbox[colback=darkgray!85!black,
		left=0mm,right=0mm,top=0mm,bottom=0mm,boxsep=1mm,toptitle=0.5mm,bottomtitle=0.5mm,
		title=\centering{A Cluny bencés apátság makettje}]{
			\includegraphics[width=1.0\linewidth]{05/cluny}
		}
	\end{wrapfigure}
	
	A román kori szerzetesi építészet jelentőségét, magas művészi értékét, gazdagságát a leghíresebb román kori szerzetesrend, a bencések \textbf{franciaország}i, Cluny-ben található kolostora példázza. A hatalmas épületegyüttesből mára mindössze a nagyobbik kereszthajó déli szárnya maradt meg, a többi rész robbantások áldozata lett. Az egykori impozáns épületegyüttesről rekonstrukciós rajzok és makettek alapján tájékozódhatunk.
	
	A bencés apátságok önálló, saját gazdasággal rendelkező kisvárosok voltak. Természetesen központjukban mindig a templom állt, amellett volt található a kolostor épülete.
	
	\paragraph{Kolostor-templom}
	Hatalmas méretű, monumentális épület volt: hosszháza 5 hajóból állt, amihez 3 hajós (!) előcsarnok csatlakozott, továbbá a templomnak 2 keresztháza volt. Az alaprajz tehát bonyolult volt, nagyon gazdag, számos kis épületrészből állt.
	
	\paragraph{A kolostor részei}
	A templomokhoz kapcsolódó kolostorok mindig azonos rend, alaprajz szerint épültek fel.
		
		\subparagraph{Kerengő}
		Azz ókeresztény átrium teréből kialakuló, elmélkedésre szolgáló, négyzet alaprajzú, árkádos folyóval körbevett udvar volt a központ.
		
		\subparagraph{Reflektórium}
		A kerenmgőből nyíló szerzetesi ebédlő, ált. egy hosszúkás terem.
		
		\subparagraph{Dormitórium}
		Szintén a kerengőből nyíló, szerzetesi alvóhely.
		
	A kolostorhoz konyha, temető, kórház, gazdasági egységek, vendégház, istálló, kert, veteményes, pékség, és további, az önálló szerzetesi élethez szükséges épületek csatlakoztak.
	
	\subsection*{Szobrászat}
	
		A szobrászati alkotások az épületekhez kapcsolódnak, domborművek (csak ritka esetben szabadon álló körplasztikák).
		\begin{compactitem}
			\item Az épület nyugati homlokzatán található domborművek.
			\item A kapu fölötti íves timpanonban lévő domborművek.
			\item Figurális oszlopfejezetek az épületbelsőben.
		\end{compactitem}
	
		\paragraph{Stílus}
		\begin{compactitem}
			\item anyaga kő
			\item a téma mindig vallásos
			\item a kompozíció egyszerű, zsúfolt
			\item nincsen valós térbeliség, térmélység, síkszerű az ábrázolás
			\item a figurák nem reálisak, nem életszerűek (,,primitívek”)
			\item a drapériaredőzés stilizált, vonalas
			\item a testtartások életszerűtlenek
			\item a mozdulatok, a gesztusok viszont kifejezőek, mindössze ezek teszik érthetővé a történeteket
			\item a tekintetek semmitmondóak
			\item a kompozíció az épülethez kapcsolódik, nincsenek szabadon álló szobrok, a fő műfaj a dombormű
			\item ragaszkodik a bibliai szöveghez, többletet nem fűz hozzá (nem célja, hogy hasson a néző érzelmeire, sem, hogy értelmezze a szöveget, sem, hogy saját gondolatot fűzzön hozzá), hanem az írástudatlan hívő számára megjeleníti, elmeséli a bibliai történeteket, ,, Biblia pauperum”, azaz a ,, Szegények Bibliája” volt a szobrász.
		\end{compactitem}
		
		\paragraph{Domborművek a homlokzatokon}
		Térmélységet nem jelentő falsík előtt jelennek meg a tömbszerűen összefogott figurák. Az előadásmód epikus, elbeszélő jellegű. A mozdulatok, testtartások egyszerűek, de ezek a kifejező gesztusok mesélik el a történetet.
		
		A figurák álltalában egyformák, felismerhetetlenek.
		
		\begin{wrapfigure}{r}{0.45\textwidth}
			\tcbox[colback=darkgray!85!black,
			left=0mm,right=0mm,top=0mm,bottom=0mm,boxsep=1mm,toptitle=0.5mm,bottomtitle=0.5mm,
			title=\centering{Maestes domini dombormű}]{
				\includegraphics[width=1.0\linewidth]{05/maestes_domini}
			}
		\end{wrapfigure}
		
		\paragraph{Domborművek a kapuk fölötti ívháromszögben}
		Leggyakoribb témája az ún. Maestas Domini, azaz ,,Fenséges Úr” = János apostolnak a Jelenések könyvében vagy más néven az Apokalipszis-ben található víziója. Leghíresebb példa:
		Arles, Saint Trophim (árli szen trofim) ívtimpanonja Franciaország.
		
		A téma kialakulásának oka az lehetett, hogy 1000 körül - mivel kerek évszám - igen erős volt a végidő-várás, így nem csoda, hogy egy azzal összefüggő, arra utaló látomás készült leggyakrabban a templomok bejárata fölé.
		
		\paragraph{Figurális oszlopfejezetek}
		Az érett román korban eltűnnek a kockafejezetek, helyettük az épületbelsőben gyakran plasztikusan kifaragott figurális kompozíciók jelennek meg az oszlopok tetején. Ezek általában szörnyeket, ijesztő állatfigurákat, a büntetésre utaló bibliai történetek bukott, pokolra jutott alakjait mutatják.
		
	\clearpage
	
	\subsection*{Magyar román kor}
	
	\begin{figure}[H]
		\centering
		\begin{minipage}{0.47\textwidth}
			\tcbox[colback=darkgray!85!black,
			left=0mm,right=0mm,top=0mm,bottom=0mm,boxsep=1mm,toptitle=0.5mm,bottomtitle=0.5mm,
			title=\centering{Jáki templom}]{
				\includegraphics[width=1.0\linewidth]{05/jak}
			}
		\end{minipage}
		\hfill
		\begin{minipage}{0.47\textwidth}
			
			\tcbox[colback=darkgray!85!black,
			left=0mm,right=0mm,top=0mm,bottom=0mm,boxsep=1mm,toptitle=0.5mm,bottomtitle=0.5mm,
			title=\centering{Pécsi székesegyház}]{
				\includegraphics[width=1.0\linewidth]{05/pecs_szekesegyhaz}
			}
		\end{minipage}
	

		\begin{minipage}{0.7\textwidth}
			
			\tcbox[colback=darkgray!85!black,
			left=0mm,right=0mm,top=0mm,bottom=0mm,boxsep=1mm,toptitle=0.5mm,bottomtitle=0.5mm,
			title=\centering{Szent István szarkofág}]{
				\includegraphics[width=1.0\linewidth]{05/szent_istvan_szarkofag}
			}
		\end{minipage}
	\end{figure}
	
	

\cleardoublepage


\section{A románkori freskófestészet, kódexfestészet és alapozási technikák}

\begin{center}
	\begin{longtable}{ | p{0.25\textwidth} | p{0.75\textwidth} | }
		
		\hline
		\multicolumn{2}{|c|}{\textbf{A tétel adatai}}
		\\ \hline
		
		\hline
		\centering{Tétel teljes címe}
		&
		Milyen hasonlóságokat és különbségeket talál a román kori freskófestészet és a kódexfestészet között tartalmilag, formailag és technikailag? Mutassa be a középkorban elterjedt alapozási technikákat, pigmenteket, kötőanyagokat!
		\\ \hline
		
	\end{longtable}
\end{center}

\subsection*{Románkori festészet}

A román kori festészet kezdetét az építészetével és a szobrászatával együtt az ezredfordulóra teszik. A falfestészetből nagyon kevés és rossz állapotú emlékanyag maradt az utókorra. 

\subsection*{Kódexfestészet}

\begin{wrapfigure}{r}{0.4\textwidth}
	\tcbox[colback=darkgray!85!black,
	left=0mm,right=0mm,top=0mm,bottom=0mm,boxsep=1mm,toptitle=0.5mm,bottomtitle=0.5mm,
	title=\centering{Oroszlán Henrik evangeliáruma, 12. század}]{
		\includegraphics[width=1.0\linewidth]{05/kodex}
	}
\end{wrapfigure}

Jelentősen megnövekedett a művészetpártoló világiak száma, az uralkodó mellett már a nemesek is rendeltek értékes kódexeket.

A középkori könyvfestészet leggyakoribb témája Krisztus élete, csodái és szenvedéstörténete. Legtöbbször az evangeliárumokat díszítették ezekkel, melyek teljes egészében vagy részleteikben tartalmazták az evangéliumokat. Krisztus szenvedéseit általában kevésbé hangsúlyozták, pl. az Ottó-korban hagyományosan Krisztus csodáira helyezték a hangsúlyt. A kódexekben sosem törekedtek Krisztus teljes életének illusztrálására, és a megbízók a Megváltó életének pozitív eseményeit akarták látni, hatásos, reprezentatív formában. Esztétikailag "eladhatóvá" akarták tenni az üdvtörténetet. Ez arra utal, hogy az egyház nagy súlyt fektetett a -mai szóval élve- propagandára.

\subsection*{Románkori freskófestészet}

	\begin{wrapfigure}{r}{0.5\textwidth}
	\tcbox[colback=darkgray!85!black,
	left=0mm,right=0mm,top=0mm,bottom=0mm,boxsep=1mm,toptitle=0.5mm,bottomtitle=0.5mm,
	title=\centering{Románkori Maestes domini témájú freskó}]{
		\includegraphics[width=1.0\linewidth]{05/fresko}
	}
\end{wrapfigure}

A korai román-kori festészetben gyakran alkalmazták a könyvfestészet kompozíciós megoldásait és motívumait. A kéziratok gyorsan terjedtek, a mérvadó központokban legkésőbb a 11. században már ismerték a fontosabb kódexeket. Az érett középkorban már nemzetközi stílusról beszélhetünk, a művészek vándoroltak, különböző uralkodók és főpapok megrendelésére dolgoztak.

A román-kori festészet egy sajátos formája a cakkos stílus, egy különös és rövid életű stílusirányzat. Jellemző rá, hogy a figurák ruhájának szegélye szokatlanul éles szögben törik meg, a formák szinte manierista stilizálása átmenetet teremt a gótika festészetéhez.

\subsection*{A freskó- és kódexfestészet összehasonlítása}

\paragraph{Különbségek}
A kódexfestészetben a könyvekbe kerülő miniatúrák kisebb méretüknél fogva drágább pigmentek felhasználásával is készülhetett, így azok színei élénkebbek.

A kódexekre jellemzőek voltak az iniciálék, a díszes betűk használata.

A méretüknél fogva részletgazdagságukban is különböztek.

\paragraph{Hasonlóságok}

Mind a kódexfestészetnek, mind a freskófestészeneklt a korra jellemző apátságok, szerzetesrendek kolostorai voltak a központjai.

Mind a kettőre jellemző volt a dekoratív színhatás, síkszerű ábrázolásmód, a narratív, elbeszélő előadásmód (ugyan ez jellemző volt a domborművekre is).

Mindkettő esetében meghatározó eszmei háttér a kereszténység, témáik bibliai történetek, szentek, vértanúk élete volt.

Mindkettőre jellemzőek voltak a díszítő motívumok, és sötét kontúrok.

\paragraph{Alapanyagok}

	A kódexfestészethez tojástemperát használtak, ásványi festékeket és növényi olajokat.
	
	A freskó esetében mészálló festékre volt szükség: fémoxidok, földfestékek.

\subsection*{A freskó}

	\subsubsection{Alapanyagok}
	
	Gipsz, mészhabarcs alapozás.
	
	Csak mészálló festékek használhatók, elsősorban fémoxidok, földfestékek, ez behatárolja a rendelkezésre álló színskálát.
	
	\subsubsection{Menete}
	
	\begin{itemize}
		\item A freskófestészet meghatározó törvénye, hogy a vakolat és a mészfesték viszonylag hamar megköt, tehát 6–8 óra alatt a munkát be kell fejezni, utólagos finomításokra már nincs mód.
		
		\item Nagyobb felületű képek készítése tehát csak kisebb, egy nap alatt elkészíthető adagokban (giornate) lehetséges, egyszerre mindig csak néhány négyzetméternyi felületet készítenek elő, majd festenek meg.
		
		\item A karton vázlat: Az előbbiekből következik, hogy a munka megkezdésekor már tökéletesen kiforrott elképzeléssel, tervekkel kell rendelkeznie a művésznek, és ezt minél gyorsabban meg kell tudnia valósítani. Ennek érdekében legtöbbször eredeti (1:1-es méretű) nagyságú vázlatot, kartont készítenek, amelyet majd a helyszínen a festés megkezdése előtt „pauzálnak”, a felületre átmásolnak. Ennek különösen nagy jelentősége van akkor, ha a befestendő felület geometriailag bonyolult formájú, például egy kupola belső felülete.
	\end{itemize}

	\paragraph{Lépései}
	
	\begin{enumerate}
		\item Alapozás: A freskó hordozója a vakolat, ennek minősége határozza meg az egész mű tartósságát.
		\begin{compactitem}
			\item Először le kell verni tégláig a régi, alkalmatlan vakolatot a falról. Majd kikaparjuk a fugákat. Az alapozás a csupasz téglafal benedvesítésével kezdődik.
			
			\item Maj erre kerül több rétegben a friss vakolat.
		\end{compactitem}
	
		\item Festés: A festés a harmadik réteg vakolatra kerül, annak felhordása után kb. egy óra múlva kell megkezdeni, és 6-8 óra alatt be is kell fejezni. 
		
		\item Ezután a nedves falból, illetve a vakolatból kifelé szivárgó meszes víz a kép felszínén szétterül, és a levegővel érintkezve vékony mészpáncélt alkot.
		
		\item A javítás ezután csak a vakolat teljes leverése és újraalapozása után lehetséges. Teljes száradás után a színek áttetsző mészkőkristályokba dermedve szinte örök életűek.
	\end{enumerate}

\paragraph{Története}

	A legkorábbi ismert al fresco készült falfestmények kb. i. e. 1500-ból származnak, Kréta szigetén a knósszoszi palotában találhatók, de Pompejiből is maradtak fenn kitűnő állapotban lévő freskók. Fénykora az itáliai reneszánszban volt. A freskó technikája az évezredek során szinte semmit se változott, mivel bebizonyította rendkívüli tartósságát.


\cleardoublepage

\chapter{A gótika építészete és az olajfesték megjelenése} % Introduction
\label{ch:6_gotika_epetiszet}

\section{A gótika építészete}

\begin{center}
	\begin{longtable}{ | p{0.25\textwidth} | p{0.75\textwidth} | }
		
		\hline
		\multicolumn{2}{|c|}{\textbf{A tétel adatai}}
		\\ \hline
		
		\hline
		Tétel teljes címe
		&
		Mutassa be a gótika építészetét! Soroljon fel, és röviden jellemezzen híres katedrálisokat és hazai műemlékeket!
		\\ \hline
		
		Jegyzetek
		&
		\begin{compactitem}
			\item A katolikus egyház szerepe.
			\item A gótikus építészeti stílus elterjedése, jellemzői Európában és Magyarországon.
			\item A templomépítészet: a katedrálisok alaprajza, részei, felépítése.
		\end{compactitem}
		\\\hline
		
	\end{longtable}
\end{center}

\cleardoublepage


\section{A tojástempera, jelentősége a klasszikus olajkép technika kialakulásában és a szárnyasoltárok}

\begin{center}
	\begin{longtable}{ | p{0.25\textwidth} | p{0.75\textwidth} | }
		
		\hline
		\multicolumn{2}{|c|}{\textbf{A tétel adatai}}
		\\ \hline
		
		\hline
		Tétel teljes címe 
		&
		Beszéljen a tojástempera technikáról és jelentőségéről a klasszikus olajkép technika kialakulásában! Jellemzően milyen alapra és milyen pigmentek, kötőanyagok felhasználásával készültek a szárnyasoltárok? Mutasson be egy szárnyasoltárt részletesebben!
		\\ \hline
		
	\end{longtable}
\end{center}

\cleardoublepage

\chapter{A gótikus szobrászat, festészet, kézművesség és ólomüveg-technika} % Introduction
\label{ch:7_gotikus_alkotasok}

\section{A gótikus szobrászat, festészet és kézművesség alkotásai}

\begin{center}
	\begin{longtable}{ | p{0.25\textwidth} | p{0.75\textwidth} | }
		
		\hline
		\multicolumn{2}{|c|}{\textbf{A tétel adatai}}
		\\ \hline
		
		\hline
		Tétel teljes címe
		&
		Mutassa be a gótikus szobrászat, festészet és kézművesség alkotásait!
		\\ \hline
		
		Jegyzetek &
		\begin{compactitem}
			\item A gótikus művészet álltalános jellemzői, fejlődése, stílusjegyei.
			\item A gótikus szobrászat, festészet az északi országokban, Itáliában és Magyarországon.
			\item Az iparművészet kiemelkedő alkotásai - üvegablakok, fémművesség.
		\end{compactitem}
		\\\hline
		
	\end{longtable}
\end{center}

\cleardoublepage


\section{Az ólomüveg-technika}

\begin{center}
	\begin{longtable}{ | p{0.25\textwidth} | p{0.75\textwidth} | }
		
		\hline
		\multicolumn{2}{|c|}{\textbf{A tétel adatai}}
		\\ \hline
		
		\hline
		Tétel teljes címe 
		&
		Miért az ólomüveg-technika válik a gótika meghatározó stílusává? Soroljon fel jelentős gótikus ólomüveg ablak alkotásokat! Mely korokban készültek még jelentős ólomüveg ablakok?
		\\ \hline
		
	\end{longtable}
\end{center}

\cleardoublepage

\chapter{Az itáliai reneszánsz művészet} % Introduction
\label{ch:8_reneszansz}

\section{Az itáliai reneszánsz építészet, szobrászat és festészet}

\begin{center}
	\begin{longtable}{ | p{0.25\textwidth} | p{0.75\textwidth} | }
		
		\hline
		\multicolumn{2}{|c|}{\textbf{A tétel adatai}}
		\\ \hline
		
		\hline
		Tétel teljes címe
		& 
		Mutassa be az itáliai reneszánsz építészet, szobrászat, festészet stílusjegyeit, jellemzőit, alkotóit és műveit!
		\\ \hline
		
		Jegyzetek &
		\begin{compactitem}
			\item Az olasz reneszánsz művészet kialakulása, korszakai, jellemzői, vívmányai.
			\item Az olasz reneszánsz épülettípusai, a szobrászat stílusjegyei az egyes korszakokban.
			\item A festészet műfajai, technikái; festők és alkotások.
		\end{compactitem}
		\\\hline
		
	\end{longtable}
\end{center}

\cleardoublepage


\section{Az itáliai reneszánsz freskó festészet}

\begin{center}
	\begin{longtable}{ | p{0.25\textwidth} | p{0.75\textwidth} | }
		
		\hline
		\multicolumn{2}{|c|}{\textbf{A tétel adatai}}
		\\ \hline
		
		\hline
		Tétel teljes címe 
		&
		Mi jellemzi az itáliai reneszánsz freskó festészetet? Mutassa be a freskókészítés munkamenetét! Milyen festékek és kötőanyagok használhatóak a freskó festészeti technikánál?
		\\ \hline
		
	\end{longtable}
\end{center}

\cleardoublepage

\chapter{A reneszánsz művészet Itálián kívül} % Introduction
\label{ch:9_reneszansz_italian_kivul}

\section{A reneszánsz művészet Itálián kívüli elterjedése}

\begin{center}
	\begin{longtable}{ | p{0.25\textwidth} | p{0.75\textwidth} | }
		
		\hline
		\multicolumn{2}{|c|}{\textbf{A tétel adatai}}
		\\ \hline
		
		\hline
		Tétel teljes címe
		&
		Mutassa be a reneszánsz művészet Itálián kívüli elterjedését és jellemzőit flamand, német, francia területeken és Magyarországon!
		\\ \hline
		
		Jegyzetek &
		\begin{compactitem}
			\item Az északi reneszánsz építészet jellemzői, épülettípusai.
			\item A falmand és német festészet, grafika jellemzői, stílusjegyei, híres alkotói.
			\item A magyar királyi udvar művészete.
		\end{compactitem}
		\\\hline
		
	\end{longtable}
\end{center}

\cleardoublepage


\section{A reneszánsz kor táblaképfestészete}

\begin{center}
	\begin{longtable}{ | p{0.25\textwidth} | p{0.75\textwidth} | }
		
		\hline
		\multicolumn{2}{|c|}{\textbf{A tétel adatai}}
		\\ \hline
		
		\hline
		Tétel teljes címe 
		&
		Mely festészeti technika a meghatározó a reneszánsz kor táblaképfestészetében? Foglalja össze egy vászora festett olajkép technikai előkészületének lépéseit! Milyen vászonfajtákat ismer, amelyek alkalmasak festmény hordozónak?
		\\ \hline
		
	\end{longtable}
\end{center}

\cleardoublepage

\chapter{A barokk művészet} % Introduction
\label{ch:10_barokk_muveszet}

\section{A barokk építészet és szobrászat}

\begin{center}
	\begin{longtable}{ | p{0.25\textwidth} | p{0.75\textwidth} | }
		
		\hline
		\multicolumn{2}{|c|}{\textbf{A tétel adatai}}
		\\ \hline
		
		\hline
		\centering{Tétel teljes címe}
		&
		Mutassa be a barokk építészet és szobrászat stílusjegyeit és példáit Itáliában, a francia királyi udvarban, Ausztriában és Magyarországon!
		\\ \hline
		
		\centering{Jegyzetek}
		&
		\begin{compactitem}
			\item A barokk művészet szerepe az ellenreformáció időszakában.
			\item A barokk építészet és szobrászat stílusjegyei.
			\item Egyházi és világi barokk épülettípusai, jellemzői.
		\end{compactitem}
		\\\hline
		
	\end{longtable}
\end{center}

\cleardoublepage


\section{A kasseli barna az olajkép technikában és a lazúrozás}

\begin{center}
	\begin{longtable}{ | p{0.25\textwidth} | p{0.75\textwidth} | }
		
		\hline
		\multicolumn{2}{|c|}{\textbf{A tétel adatai}}
		\\ \hline
		
		\hline
		\centering{Tétel teljes címe}
		&
		Mia a kasseli barna, mire használjuk az olajkép technikában -- alkalmazása hogyan jelenik meg? Mutassa be a lazúrozás szerepét Caravaggio és Rubens munkáin keresztül! Milyen összetevőkből áll a lazúrlakk?
		\\ \hline
		
	\end{longtable}
\end{center}
\cleardoublepage

\chapter{A barokk és rokokó festészete} % Introduction
\label{ch:11_barokk_es_rokoko}

\section{A barokk és rokokó festészet európában}

\begin{center}
	\begin{longtable}{ | p{0.25\textwidth} | p{0.75\textwidth} | }
		
		\hline
		\multicolumn{2}{|c|}{\textbf{A tétel adatai}}
		\\ \hline
		\hline
		
		\centering{Tétel teljes címe}
		& 
		Mutassa be az egyes európai országok barokk és rokokó festészetének külböző stílusjegyeit és alkotásait!
		\\ \hline
		
		\centering{Jegyzetek}
		&
		\begin{compactitem}
			\item A barokk festészet álltalános jellemzői, szerepe.
			\item A katolikus Itália, Spanyolország, Flandria és polgári Hollandia festészete, mesterei.
			\item A magyarországi barokk festészet példái.
		\end{compactitem}
		\\\hline
		
	\end{longtable}
\end{center}

\cleardoublepage


\section{A rembrandti festészet jellemzői és olajfestés technikája}

\begin{center}
	\begin{longtable}{ | p{0.25\textwidth} | p{0.75\textwidth} | }
		
		\hline
		\multicolumn{2}{|c|}{\textbf{A tétel adatai}}
		\\ \hline
		
		\hline
		\centering{Tétel teljes címe}
		&
		Ismertese Rembrandt egy alkotásán keresztül a rembrandti festészet jellemzőit! Milyen olajok és oldószerek használhatóak az olajfestés technikájához?
		\\ \hline
		
	\end{longtable}
\end{center}

\cleardoublepage

\chapter{A klasszicizmus művészete} % Introduction
\label{ch:12_klasszicizmus}

\section{A kalsszicista építészet, szobrászat és festészet}

\begin{center}
	\begin{longtable}{ | p{0.25\textwidth} | p{0.75\textwidth} | }
		
		\hline
		\multicolumn{2}{|c|}{\textbf{A tétel adatai}}
		\\ \hline
		\hline
		
		\centering{Tétel teljes címe}
		&
		Mutassa be a klasszicista építészet, szobrászat és festészet stílusjegyeit, alkotásait, hazai műemlékeit!
		\\ \hline
		
		\centering{Jegyzetek}
		&
		\begin{compactitem}
			\item A klasszicizmus álltalános jellemzői, társadalmi, filozófiai háttere.
			\item Az építészet stílusjegyei (Fraqnciaország, Anglia, Magyarország).
			\item A szobrászat és a festészet stílusjegyei, mesterei (Franciaország és Magyarország).
		\end{compactitem}
		\\\hline
		
	\end{longtable}
\end{center}

\cleardoublepage


\section{Klasszikus megoldások a díszletfestészetben, a táblakép és a díszletfestő technika különbségei}
\begin{center}
	\begin{longtable}{ | p{0.25\textwidth} | p{0.75\textwidth} | }
		
		\hline
		\multicolumn{2}{|c|}{\textbf{A tétel adatai}}
		\\ \hline
		\hline
		
		\centering{Tétel teljes címe} 
		&
		Milyen technikai, ábrázolási illetve szerkesztési megoldásokat vett át a díszletfestészet a klasszikus festészettől? Melyek a legjellemzőbb különbségek a táblakép és a díszletfestő technikák között?
		\\ \hline
		
	\end{longtable}
\end{center}

\cleardoublepage

\chapter{A romantika művészete és az akvarell technika} % Introduction
\label{ch:13_romantika}

\section{A romantika szobrászata, festészete és építészete}

\begin{center}
	\begin{longtable}{ | p{0.25\textwidth} | p{0.75\textwidth} | }
		
		\hline
		\multicolumn{2}{|c|}{\textbf{A tétel adatai}}
		\\ \hline
		
		\hline
		
		\centering{Tétel teljes címe}
		&
		Mutassa be a romantika szobrászatának és festészetének stílusjegyeit, a historizmus és eklektika építészetének jellemzőit, alkotásait! 
		\\ \hline
		
		\centering{Jegyzetek}
		&
		\begin{compactitem}
			\item A romantikus festészet, szobrászat stílusjegyei, irányzatai és mesterei (francia, német, angol és magyar).
			\item A historizáló és eklektikus művészet.
		\end{compactitem}
		\\\hline
		
	\end{longtable}
\end{center}

\cleardoublepage


\section{Az akvarell technika}

\begin{center}
	\begin{longtable}{ | p{0.25\textwidth} | p{0.75\textwidth} | }
		
		\hline
		\multicolumn{2}{|c|}{\textbf{A tétel adatai}}
		\\ \hline
		\hline
		
		\centering{Tétel teljes címe} 
		&
		Mely romantikus festők életművében találhatunk jelentős akvarellképeket is? Foglalja össze az akavarellfestés technikáját, anyagait! Milyen ecseteket használna akvarellképhez? Soroljon fel alkotókat, akik foglalkoptak még akvarell festészettel!
		\\ \hline
		
	\end{longtable}
\end{center}

\cleardoublepage

\chapter{Az impresszionizmus, posztimpresszionizmus és a plein air technika} % Introduction
\label{ch:14_impresszionizmus_posztimpresszionizmus}

\section{Az impresszionista és a posztimpresszionista festészet}

\begin{center}
	\begin{longtable}{ | p{0.25\textwidth} | p{0.75\textwidth} | }
		
		\hline
		\multicolumn{2}{|c|}{\textbf{A tétel adatai}}
		\\ \hline
		\hline
		
		\centering{Tétel teljes címe}
		&
		Mutassa be az impresszionista festészet stílusjegyeit, alkotóit és műveit! Ismertesse a Nagybányai művésztelep tagjainak tevékenységét!
		\\ \hline
		
		\centering{Jegyzetek}
		&
		\begin{compactitem}
			\item A 19. századi művészet célkitűzései, törekvései, plein air festészete.
			\item Az impresszionizmus és posztimpresszionizmus nagymestereinek művészete.
			\item A századvégi stílusirányzatok magyar képviselői.
		\end{compactitem}
		\\\hline
		
	\end{longtable}
\end{center}

\cleardoublepage


\section{A pein air festészet és az impresszionista festészeti megoldások}

\begin{center}
	\begin{longtable}{ | p{0.25\textwidth} | p{0.75\textwidth} | }
		
		\hline
		\multicolumn{2}{|c|}{\textbf{A tétel adatai}}
		\\ \hline
		\hline
		
		\centering{Tétel teljes címe}
		&
		Mit jelent a pein air kifejezés a festészetben? Milyen technikai újítás tette lehetővé a pein air festészetet? Milyen az impreszionista ecsethasználat? Mi az optikai színkeverés? Milyen színelméleti alapfogalmakat ismer?
		\\ \hline
		
	\end{longtable}
\end{center}

\cleardoublepage

\chapter{A szimbolizmus és a szecesszió művészete, valamint a pasztell technika} % Introduction
\label{ch:15_szimbolizmus_szecesszio}

\section{A szimbolizmus és a szecesszió Európában és Magyarországon}

\begin{center}
	\begin{longtable}{ | p{0.25\textwidth} | p{0.75\textwidth} | }
		
		\hline
		\multicolumn{2}{|c|}{\textbf{A tétel adatai}}
		\\ \hline
		\hline
		
		\centering{Tétel teljes címe}
		&
		Mutassa be a szimbolizmus festészeti irányzatát és a szecesszió építészetét, művészetét Európában és Magyarországon!
		\\ \hline
		
		\centering{Jegyzetek}
		&
		\begin{compactitem}
			\item A szimbolizmus jellemzői, törekvései.
			\item A szecesszió építészetének újításai, stílusjegyei; stílusteremtő építészek Európában és Magyarországon.
			\item A szecessziós grafika, festészet és iparművészet stílusjegyei és jeles képvisleői.
		\end{compactitem}
		\\\hline
		
	\end{longtable}
\end{center}

\cleardoublepage


\section{A pasztell festészeti technika és a fixatív}

\begin{center}
	\begin{longtable}{ | p{0.25\textwidth} | p{0.75\textwidth} | }
		
		\hline
		\multicolumn{2}{|c|}{\textbf{A tétel adatai}}
		\\ \hline
		\hline
		
		\centering{Tétel teljes címe}
		&
		Mutassa be a pasztell festészeti technikát, határozza meg a fixatív fogalmát! Nevezzen meg híres magyar és külföldi alkotókat, akik pasztellt használtak!
		\\ \hline
		
	\end{longtable}
\end{center}

\cleardoublepage

\chapter{A 20. századelejének avantgárd művészeti irányzatai I.} % Introduction
\label{ch:16_avantgard_elso_fele}

\section{A fauvizmus, az expresszionizmus és a kubizmus}

\begin{center}
	\begin{longtable}{ | p{0.25\textwidth} | p{0.75\textwidth} | }
		
		\hline
		\multicolumn{2}{|c|}{\textbf{A tétel adatai}}
		\\ \hline
		\hline
		
		\centering{Tétel teljes címe}
		&
		Mutassa be a 20. század elejének avatgárd művészeti irányzatai közül a fauvizmus, expresszionizmus, kubizmus legfőbb törekvéseit és kiemelkedő alkotóit! Ismertesse a Nyolcak csoportjának tevékenységét!
		\\ \hline
		
		\centering{Jegyzetek}
		&
		\begin{compactitem}
			\item A 20. század elejének társadalmi környezete, a művészeti irányzatok alapelvei.
			\item A 20. századi képzőművészet meghatározó alakjai.
			\item A festészet modern irányzatai, magyarországi elterjedése.
		\end{compactitem}
		\\\hline
		
	\end{longtable}
\end{center}

\cleardoublepage


\section{Az avantgárd mozgalmak és a prehisztorikus művészetek}

\begin{center}
	\begin{longtable}{ | p{0.25\textwidth} | p{0.75\textwidth} | }
		
		\hline
		\multicolumn{2}{|c|}{\textbf{A tétel adatai}}
		\\ \hline		
		\hline
		
		\centering{Tétel teljes címe}
		&
		Hogyan viszonyulnak az avantgárd mozgalmak a prehisztorikus, törzsi és primitív művészetekhez? Hogyan készülhettek a barlangfestmények? Vaszilij Kandinszkij egy alkotásán keresztül ismertesse az absztrakt festészeti kifejezésmód legfőbb jellemzőit!
		\\ \hline
		
	\end{longtable}
\end{center}

\cleardoublepage

\chapter{A 20. századelejének avantgárd művészeti irányzatai II.} % Introduction
\label{ch:17_avantgard_masodik_fele}

\section{A futurizmus, a konstruktivizmus, a dadizmus és a szürrealizmus művészeti irányzatok}

\begin{center}
	\begin{longtable}{ | p{0.25\textwidth} | p{0.75\textwidth} | }
		
		\hline
		\multicolumn{2}{|c|}{\textbf{A tétel adatai}}
		\\ \hline
		\hline
		
		\centering{Tétel teljes címe}
		&
		Mutassa be a 20. század eleji avantgárd művészeti irányzatok közül a futurizmus, a konstruktivizmus, a dadizmus és a szürrealizmus törekvéseit és kiemelkedő alkotóit! Beszéljen az abszutrakt művészet kialkulásáról!
		\\ \hline
		
		\centering{Jegyzetek}
		&
		\begin{compactitem}
			\item A 20. század elejének társadalmi környezete, a művészetek filozófiai háttere.
			\item A képzőművészet modern irányzatai, megjelenési formái és legismertebb képviselői.
			\item Az absztrakció különböző megjelenési formái.
		\end{compactitem}
		\\\hline
		
	\end{longtable}
\end{center}

\cleardoublepage


\section{A montázs és a kollázs festészeti technika}

\begin{center}
	\begin{longtable}{ | p{0.25\textwidth} | p{0.75\textwidth} | }
		
		\hline
		\multicolumn{2}{|c|}{\textbf{A tétel adatai}}
		\\ \hline
		\hline
		
		\centering{Tétel teljes címe}
		&
		Melyek a montázs és a kollázs festészeti technika sajátosságai?
		\\ \hline
		
	\end{longtable}
\end{center}

\cleardoublepage

\chapter{A 20. századi építészet, iparművészet és a street art} % Introduction
\label{ch:18_modern_epiteszet_es_iparmuveszet}

\section{A 20. századi építészet és iparművészet legfontosabb irányzatai}

\begin{center}
	\begin{longtable}{ | p{0.25\textwidth} | p{0.75\textwidth} | }
		
		\hline
		\multicolumn{2}{|c|}{\textbf{A tétel adatai}}
		\\ \hline
		\hline
		
		\centering{Tétel teljes címe}
		& 
		Mutassa be a 20. századi építészet és iparművészet legfontosabb irányzatait! Sorolja fel és jellemezze a két terület meghatározó alkotásait, tervezőit!
		\\ \hline
		
		\centering{Jegyzetek}
		&
		\begin{compactitem}
			\item A Bauhaus története és képviselői.
			\item Design áramlatok a 20. században (art deco, funkcionalizmus, organikus művészet, minimalizmus).
			\item A modern és posztmodern építészet stílusteremtő személyiségei.
		\end{compactitem}
		\\\hline
		
	\end{longtable}
\end{center}

\cleardoublepage


\section{A street art művészet}

\begin{center}
	\begin{longtable}{ | p{0.25\textwidth} | p{0.75\textwidth} | }
		
		\hline
		\multicolumn{2}{|c|}{\textbf{A tétel adatai}}
		\\ \hline
		\hline
		
		\centering{Tétel teljes címe}
		&
		Milyen képi világ jellemzi a street art munkákat? Milyen anyagokat, technikai eszközöket használ a street art művészet? Mutasson be egy ismert street art alkotót!
		\\ \hline
		
	\end{longtable}
\end{center}
\cleardoublepage

\chapter{A 20. század második felének avantgárd művészete és a csurgatásos technika}
\label{ch:19_avantgard_20_szazad_masodik_fele}

\section{A 20. század második felének avantgárd és a magyar neoavantgárd festői irányzatai}

\begin{center}
	\begin{longtable}{ | p{0.25\textwidth} | p{0.75\textwidth} | }
		
		\hline
		\multicolumn{2}{|c|}{\textbf{A tétel adatai}}
		\\ \hline
		\hline
		
		\centering{Tétel teljes címe}
		& 
		Mutassa be a 20. század második felének avantgárd és a magyar neoavantgárd művészet festői irányzatait, híres alkotásait, művészeit!
		\\ \hline
		
		\centering{Jegyzetek}
		&
		\begin{compactitem}
			\item Művészeti válaszok a II. világháború utáni társadalmi jelenségekre.
			\item Az absztrakt expresszionizmus, pop art, hiperrealista festészet külföldi és magyar képviselői.
		\end{compactitem}
		\\\hline
		
	\end{longtable}
\end{center}

\cleardoublepage


\section{A csurgatásos festéstechnika és Jackson Pollock}

\begin{center}
	\begin{longtable}{ | p{0.25\textwidth} | p{0.75\textwidth} | }
		
		\hline
		\multicolumn{2}{|c|}{\textbf{A tétel adatai}}
		\\ \hline
		\hline
		
		\centering{Tétel teljes címe} 
		&
		Határozza meg a csurgatásos festéstechnika (Jackson Pollock) lényegét! Hasonlítsa össze ezt a hard edge festészetének módszerével, nevezzen meg alkotókat! Milyen tulajdonságokkal bírnak az akril alapú festékek?
		\\ \hline
		
	\end{longtable}
\end{center}

\cleardoublepage

\chapter{Újabb képzőművészeti műfajok a 20. század második felében és kortárs alkotók} % Introduction
\label{ch:20_uj_iranyzatok_20_szazad_vegen}

\section{Újabb képzőművészeti műfajok, irányzatok a 20. század második felében}

\begin{center}
	\begin{longtable}{ | p{0.25\textwidth} | p{0.75\textwidth} | }
		
		\hline
		\multicolumn{2}{|c|}{\textbf{A tétel adatai}}
		\\ \hline
		\hline
		
		\centering{Tétel teljes címe}
		&
		Mutassa be a 20. század második felében megjelenő újabb képzőművészeti műfajokat, irányzatokat!
		\\ \hline
		
		\centering{Jegyzetek}
		&
		\begin{compactitem}
			\item A korszak új művészeti megnyilvánulásai Európában és az Egyesült Államokban és Magyarországon.
			\item A konceptuális művészet törekvései, fluxus, performance, akcióművészet, az installáció és médiaművészet alkotásai, művészei.
		\end{compactitem}
		\\\hline
		
	\end{longtable}
\end{center}

\cleardoublepage


\section{Kortárs művészek és új technikai lehetőségek a kortárs festészetben}

\begin{center}
	\begin{longtable}{ | p{0.25\textwidth} | p{0.75\textwidth} | }
		
		\hline
		\multicolumn{2}{|c|}{\textbf{A tétel adatai}}
		\\ \hline
		\hline
		
		\centering{Tétel teljes címe}
		&
		Soroljon fel legalább 3 kortárs alkotót, akik meghatározóak a jelenkori magyar festészetben! Beszéljen művészetükről, festészetük sajátosságairól! Melyek azok az új technikai lehetőségek, eljárások, amelyeket a kortárs festészet meghatározó módon használ fel?
		\\ \hline
		
	\end{longtable}
\end{center}

\cleardoublepage

\chapter{Források} % Introduction
\label{ch:forrasok}

\begin{itemize}
	\item Garami Gréta: Művészettörténet 2 (jegyzet)
	\item Petrőcz Evelin: Művészettörténet képanyagok
	\item Dittrich Renáta: tételek
	\item Az ókori egyiptom
	\begin{compactitem}
		\item \url{https://fett.gportal.hu/gindex.php?pg=4090439}
		\item \url{https://hu.khanacademy.org/humanities/kozepiskolai-tortenelem/x3c94c9499459dcd5:okor/x3c94c9499459dcd5:az-okori-egyiptom/a/egypt-article}
		\item \url{https://www.nkp.hu/tankonyv/tortenelem_5_nat2020/lecke_02_003}
	\end{compactitem}
	
\end{itemize}
\cleardoublepage

\end{document}
