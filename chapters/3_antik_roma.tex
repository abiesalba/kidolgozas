\chapter{Az antik Róma művészete} % Introduction
\label{ch:3_antik_roma}

\section{Az antik Róma korszakai, kultúrája, építészete, szobrászata és festészete}

\begin{center}
	\begin{longtable}{ | p{0.25\textwidth} | p{0.75\textwidth} | }
		
		\hline
		\multicolumn{2}{|c|}{\textbf{A tétel adatai}}
		\\ \hline
		
		\hline
		Tétel teljes címe
		& 
		Mutassa be az antik Róma korszakait, kultúráját, építészetének, szobrászatának és festészetének jellegzetésseit! Soroljon fel és jellemezzen néhány fennmaradt római építményt és alkotást!
		\\ \hline
		
		Jegyzetek &
		\begin{compactitem}
			\item Az ókori Római Birodalom társadalmi felépítése, a római művészet korszakai és jellemzői.
			\item Az építészet jellemzői, építmény-típusai, a szobrászat és festészet emlékei.
			\item Pannónia provincia római kori maradványai.
		\end{compactitem}
		\\\hline
		
	\end{longtable}
\end{center}

\cleardoublepage


\section{A római kori festészet alapvető jellemzői}

\begin{center}
	\begin{longtable}{ | p{0.25\textwidth} | p{0.75\textwidth} | }
		
		\hline
		\multicolumn{2}{|c|}{\textbf{A tétel adatai}}
		\\ \hline
		
		\hline
		Tétel teljes címe 
		&
		Egy konkrét alkotáson keresztül ismertesse a római kori festészet alapvető jellemzőit! Beszéljen a Pompeii-ben feltárt falfestmények alapján a négy meghatározó festészeti stílusáról! Mi az enkausztika?
		\\ \hline
		
	\end{longtable}
\end{center}