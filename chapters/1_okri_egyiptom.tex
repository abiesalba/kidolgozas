\chapter{Az ókori Egyiptom művészete} % Introduction
\label{ch:1_okri_egyiptom}

\section{Az ókori egyiptom társadalmi felépítése, hiedelemvilága, művészeti korszakai és emlékei}

\begin{center}
	\begin{longtable}{ | p{0.25\textwidth} | p{0.75\textwidth} | }
		
		\hline
		\multicolumn{2}{|c|}{\textbf{A tétel adatai}}
		\\ \hline
		
		\hline
		Tétel teljes címe & Mutassa be az ókori Egyiptom társadalmi felépítését, hiedelemvilágát! Ismertese az ókori egyiptomi művészet korszakait, az építészet, szobrászat és festészet ránk maradt emlékeinek jellegzetességeit!
		\\ \hline
		
		Jegyzetek &
		\begin{compactitem}
			\item Az ókori egyiptomi művészet korszakai, földrajzi, társadalmi környezete.
			\item A sír- és templomépítészet típusai, felépítése, jellmzői.
			\item A szobrászat, a festészet és tárgykultúra jellemzői és stílusjegyei.
		\end{compactitem}
		\\\hline
				
	\end{longtable}
\end{center}

\cleardoublepage

\section{Az ókor meghatározó festészeti techinkái: az egyiptomi falkép}

\begin{center}
	\begin{longtable}{ | m{0.25\textwidth} | p{0.75\textwidth} | }
		
		\hline
		\multicolumn{2}{|c|}{\textbf{A tétel adatai}}
		\\ \hline
		
		\hline
		Tétel teljes címe	
		 &
		 Melyek a meghatározó festészeti technikák az ókorban? Fejtse ki, miként alakult egy egyiptomi falkép elkészítésének munkamenete, milyen alapozást, pigmenteket és kötőanyagot használtak?
		\\ \hline
		
	\end{longtable}
\end{center}