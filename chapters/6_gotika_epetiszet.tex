\chapter{A gótika építészete és az olajfesték megjelenése} % Introduction
\label{ch:6_gotika_epetiszet}

\section{A gótika építészete}

\begin{center}
	\begin{longtable}{ | p{0.25\textwidth} | p{0.75\textwidth} | }
		
		\hline
		\multicolumn{2}{|c|}{\textbf{A tétel adatai}}
		\\ \hline
		
		\hline
		Tétel teljes címe
		&
		Mutassa be a gótika építészetét! Soroljon fel, és röviden jellemezzen híres katedrálisokat és hazai műemlékeket!
		\\ \hline
		
		Jegyzetek
		&
		\begin{compactitem}
			\item A katolikus egyház szerepe.
			\item A gótikus építészeti stílus elterjedése, jellemzői Európában és Magyarországon.
			\item A templomépítészet: a katedrálisok alaprajza, részei, felépítése.
		\end{compactitem}
		\\\hline
		
	\end{longtable}
\end{center}

\cleardoublepage


\section{A tojástempera, jelentősége a klasszikus olajkép technika kialakulásában és a szárnyasoltárok}

\begin{center}
	\begin{longtable}{ | p{0.25\textwidth} | p{0.75\textwidth} | }
		
		\hline
		\multicolumn{2}{|c|}{\textbf{A tétel adatai}}
		\\ \hline
		
		\hline
		Tétel teljes címe 
		&
		Beszéljen a tojástempera technikáról és jelentőségéről a klasszikus olajkép technika kialakulásában! Jellemzően milyen alapra és milyen pigmentek, kötőanyagok felhasználásával készültek a szárnyasoltárok? Mutasson be egy szárnyasoltárt részletesebben!
		\\ \hline
		
	\end{longtable}
\end{center}
