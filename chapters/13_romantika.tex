\chapter{A romantika művészete és az akvarell technika} % Introduction
\label{ch:13_romantika}

\section{A romantika szobrászata, festészete és építészete}

\begin{center}
	\begin{longtable}{ | p{0.25\textwidth} | p{0.75\textwidth} | }
		
		\hline
		\multicolumn{2}{|c|}{\textbf{A tétel adatai}}
		\\ \hline
		
		\hline
		
		\centering{Tétel teljes címe}
		&
		Mutassa be a romantika szobrászatának és festészetének stílusjegyeit, a historizmus és eklektika építészetének jellemzőit, alkotásait! 
		\\ \hline
		
		\centering{Jegyzetek}
		&
		\begin{compactitem}
			\item A romantikus festészet, szobrászat stílusjegyei, irányzatai és mesterei (francia, német, angol és magyar).
			\item A historizáló és eklektikus művészet.
		\end{compactitem}
		\\\hline
		
	\end{longtable}
\end{center}

\cleardoublepage


\section{Az akvarell technika}

\begin{center}
	\begin{longtable}{ | p{0.25\textwidth} | p{0.75\textwidth} | }
		
		\hline
		\multicolumn{2}{|c|}{\textbf{A tétel adatai}}
		\\ \hline
		\hline
		
		\centering{Tétel teljes címe} 
		&
		Mely romantikus festők életművében találhatunk jelentős akvarellképeket is? Foglalja össze az akavarellfestés technikáját, anyagait! Milyen ecseteket használna akvarellképhez? Soroljon fel alkotókat, akik foglalkoptak még akvarell festészettel!
		\\ \hline
		
	\end{longtable}
\end{center}
