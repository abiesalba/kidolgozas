\chapter{A 20. századelejének avantgárd művészeti irányzatai II.} % Introduction
\label{ch:17_avantgard_masodik_fele}

\section{A futurizmus, a konstruktivizmus, a dadizmus és a szürrealizmus művészeti irányzatok}

\begin{center}
	\begin{longtable}{ | p{0.25\textwidth} | p{0.75\textwidth} | }
		
		\hline
		\multicolumn{2}{|c|}{\textbf{A tétel adatai}}
		\\ \hline
		\hline
		
		\centering{Tétel teljes címe}
		&
		Mutassa be a 20. század eleji avantgárd művészeti irányzatok közül a futurizmus, a konstruktivizmus, a dadizmus és a szürrealizmus törekvéseit és kiemelkedő alkotóit! Beszéljen az abszutrakt művészet kialkulásáról!
		\\ \hline
		
		\centering{Jegyzetek}
		&
		\begin{compactitem}
			\item A 20. század elejének társadalmi környezete, a művészetek filozófiai háttere.
			\item A képzőművészet modern irányzatai, megjelenési formái és legismertebb képviselői.
			\item Az absztrakció különböző megjelenési formái.
		\end{compactitem}
		\\\hline
		
	\end{longtable}
\end{center}

\cleardoublepage


\section{A montázs és a kollázs festészeti technika}

\begin{center}
	\begin{longtable}{ | p{0.25\textwidth} | p{0.75\textwidth} | }
		
		\hline
		\multicolumn{2}{|c|}{\textbf{A tétel adatai}}
		\\ \hline
		\hline
		
		\centering{Tétel teljes címe}
		&
		Melyek a montázs és a kollázs festészeti technika sajátosságai?
		\\ \hline
		
	\end{longtable}
\end{center}
