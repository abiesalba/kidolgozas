\chapter{A 20. századi építészet, iparművészet és a street art} % Introduction
\label{ch:18_modern_epiteszet_es_iparmuveszet}

\section{A 20. századi építészet és iparművészet legfontosabb irányzatai}

\begin{center}
	\begin{longtable}{ | p{0.25\textwidth} | p{0.75\textwidth} | }
		
		\hline
		\multicolumn{2}{|c|}{\textbf{A tétel adatai}}
		\\ \hline
		\hline
		
		\centering{Tétel teljes címe}
		& 
		Mutassa be a 20. századi építészet és iparművészet legfontosabb irányzatait! Sorolja fel és jellemezze a két terület meghatározó alkotásait, tervezőit!
		\\ \hline
		
		\centering{Jegyzetek}
		&
		\begin{compactitem}
			\item A Bauhaus története és képviselői.
			\item Design áramlatok a 20. században (art deco, funkcionalizmus, organikus művészet, minimalizmus).
			\item A modern és posztmodern építészet stílusteremtő személyiségei.
		\end{compactitem}
		\\\hline
		
	\end{longtable}
\end{center}

\cleardoublepage


\section{A street art művészet}

\begin{center}
	\begin{longtable}{ | p{0.25\textwidth} | p{0.75\textwidth} | }
		
		\hline
		\multicolumn{2}{|c|}{\textbf{A tétel adatai}}
		\\ \hline
		\hline
		
		\centering{Tétel teljes címe}
		&
		Milyen képi világ jellemzi a street art munkákat? Milyen anyagokat, technikai eszközöket használ a street art művészet? Mutasson be egy ismert street art alkotót!
		\\ \hline
		
	\end{longtable}
\end{center}