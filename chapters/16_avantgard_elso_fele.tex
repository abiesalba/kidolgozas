\chapter{A 20. századelejének avantgárd művészeti irányzatai I.} % Introduction
\label{ch:16_avantgard_elso_fele}

\section{A fauvizmus, az expresszionizmus és a kubizmus}

\begin{center}
	\begin{longtable}{ | p{0.25\textwidth} | p{0.75\textwidth} | }
		
		\hline
		\multicolumn{2}{|c|}{\textbf{A tétel adatai}}
		\\ \hline
		\hline
		
		\centering{Tétel teljes címe}
		&
		Mutassa be a 20. század elejének avatgárd művészeti irányzatai közül a fauvizmus, expresszionizmus, kubizmus legfőbb törekvéseit és kiemelkedő alkotóit! Ismertesse a Nyolcak csoportjának tevékenységét!
		\\ \hline
		
		\centering{Jegyzetek}
		&
		\begin{compactitem}
			\item A 20. század elejének társadalmi környezete, a művészeti irányzatok alapelvei.
			\item A 20. századi képzőművészet meghatározó alakjai.
			\item A festészet modern irányzatai, magyarországi elterjedése.
		\end{compactitem}
		\\\hline
		
	\end{longtable}
\end{center}

\cleardoublepage


\section{Az avantgárd mozgalmak és a prehisztorikus művészetek}

\begin{center}
	\begin{longtable}{ | p{0.25\textwidth} | p{0.75\textwidth} | }
		
		\hline
		\multicolumn{2}{|c|}{\textbf{A tétel adatai}}
		\\ \hline		
		\hline
		
		\centering{Tétel teljes címe}
		&
		Hogyan viszonyulnak az avantgárd mozgalmak a prehisztorikus, törzsi és primitív művészetekhez? Hogyan készülhettek a barlangfestmények? Vaszilij Kandinszkij egy alkotásán keresztül ismertesse az absztrakt festészeti kifejezésmód legfőbb jellemzőit!
		\\ \hline
		
	\end{longtable}
\end{center}
