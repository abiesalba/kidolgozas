\chapter{A gótikus szobrászat, festészet, kézművesség és ólomüveg-technika} % Introduction
\label{ch:7_gotikus_alkotasok}

\section{A gótikus szobrászat, festészet és kézművesség alkotásai}

\begin{center}
	\begin{longtable}{ | p{0.25\textwidth} | p{0.75\textwidth} | }
		
		\hline
		\multicolumn{2}{|c|}{\textbf{A tétel adatai}}
		\\ \hline
		
		\hline
		Tétel teljes címe
		&
		Mutassa be a gótikus szobrászat, festészet és kézművesség alkotásait!
		\\ \hline
		
		Jegyzetek &
		\begin{compactitem}
			\item A gótikus művészet álltalános jellemzői, fejlődése, stílusjegyei.
			\item A gótikus szobrászat, festészet az északi országokban, Itáliában és Magyarországon.
			\item Az iparművészet kiemelkedő alkotásai - üvegablakok, fémművesség.
		\end{compactitem}
		\\\hline
		
	\end{longtable}
\end{center}

\cleardoublepage


\section{Az ólomüveg-technika}

\begin{center}
	\begin{longtable}{ | p{0.25\textwidth} | p{0.75\textwidth} | }
		
		\hline
		\multicolumn{2}{|c|}{\textbf{A tétel adatai}}
		\\ \hline
		
		\hline
		Tétel teljes címe 
		&
		Miért az ólomüveg-technika válik a gótika meghatározó stílusává? Soroljon fel jelentős gótikus ólomüveg ablak alkotásokat! Mely korokban készültek még jelentős ólomüveg ablakok?
		\\ \hline
		
	\end{longtable}
\end{center}
