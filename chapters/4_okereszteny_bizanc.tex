\chapter{Az ókeresztény és a bizánci művészet} % Introduction
\label{ch:4_okereszteny_bizanc}

\section{Az ókeresztény és a bizánci építészet, festészet és díszítőművészet}

\begin{center}
	\begin{longtable}{ | p{0.25\textwidth} | p{0.75\textwidth} | }
		
		\hline
		\multicolumn{2}{|c|}{\textbf{A tétel adatai}}
		\\ \hline
		
		\hline
		Tétel teljes címe
		&
		Mutassa be az ókeresztény és a bizánci művészetet - az építészet és a festészet, valamint a díszítőművészet stílusjegyeit, ismert alkotásait!
		\\ \hline
		
		Jegyzetek
		&
		\begin{compactitem}
			\item Az ókeresztény és bizánci építészet - a bazilika és a bizánci templom felépítése.
			\item Az ókeresztény szobrászat stílusjegyei és jelképei.
			\item A bizánci mozaik, az ikonfestészet és a kézművesség jellemzői.
		\end{compactitem}
		\\\hline
		
	\end{longtable}
\end{center}

\cleardoublepage


\section{Az ikonfestészet}

\begin{center}
	\begin{longtable}{ | p{0.25\textwidth} | p{0.75\textwidth} | }
		
		\hline
		\multicolumn{2}{|c|}{\textbf{A tétel adatai}}
		\\ \hline
		
		\hline
		Tétel teljes címe 
		&
		Milyen szabályrendszer határozza meg az ikonok képi világát? Mutassa be az ikonfestés technikáját, lépéseit! Milyen anyagok szükségesek egy ikon festéséhez? Milyen eszközök szükségesek az aranyozáshoz?
		\\ \hline
		
	\end{longtable}
\end{center}
