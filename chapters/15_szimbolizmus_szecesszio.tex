\chapter{A szimbolizmus és a szecesszió művészete, valamint a pasztell technika} % Introduction
\label{ch:15_szimbolizmus_szecesszio}

\section{A szimbolizmus és a szecesszió Európában és Magyarországon}

\begin{center}
	\begin{longtable}{ | p{0.25\textwidth} | p{0.75\textwidth} | }
		
		\hline
		\multicolumn{2}{|c|}{\textbf{A tétel adatai}}
		\\ \hline
		\hline
		
		\centering{Tétel teljes címe}
		&
		Mutassa be a szimbolizmus festészeti irányzatát és a szecesszió építészetét, művészetét Európában és Magyarországon!
		\\ \hline
		
		\centering{Jegyzetek}
		&
		\begin{compactitem}
			\item A szimbolizmus jellemzői, törekvései.
			\item A szecesszió építészetének újításai, stílusjegyei; stílusteremtő építészek Európában és Magyarországon.
			\item A szecessziós grafika, festészet és iparművészet stílusjegyei és jeles képvisleői.
		\end{compactitem}
		\\\hline
		
	\end{longtable}
\end{center}

\cleardoublepage


\section{A pasztell festészeti technika és a fixatív}

\begin{center}
	\begin{longtable}{ | p{0.25\textwidth} | p{0.75\textwidth} | }
		
		\hline
		\multicolumn{2}{|c|}{\textbf{A tétel adatai}}
		\\ \hline
		\hline
		
		\centering{Tétel teljes címe}
		&
		Mutassa be a pasztell festészeti technikát, határozza meg a fixatív fogalmát! Nevezzen meg híres magyar és külföldi alkotókat, akik pasztellt használtak!
		\\ \hline
		
	\end{longtable}
\end{center}
