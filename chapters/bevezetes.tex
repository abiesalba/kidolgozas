\chapter{Bevezetés} % Introduction
\label{ch:bevezetes}

\section{A vizsga időpontja}

\begin{compactitem}
	\item 2022. Június 9., csütörtök
	\begin{compactitem} 
		\item 8:00-tól: a portfólió bemutatása (a vizsga kezdete előtt meg kell jelenni).
		\item 10:45-től: vizsgaremek bemutatása.
	\end{compactitem}
	
	\item 2022. Június 10., péntek
	\begin{compactitem}
		\item 8:00-tól: szóbeli vizsgák (a vizsga kezdete előtt meg kell jelenni).
		\item Az iskola részéről vizsgabizottsági tagok: Balogh Ede, Haramati Zsófia, Sárközi Antal.
	\end{compactitem}

	\item Megjegyzések
	\begin{compactitem}
		\item Alkalomhoz illő öltözet elvárt.
		\item A vizsga együtt lesz a nappali OKJ-s festőkkel (összesen 20 fő együtt).
	\end{compactitem}
\end{compactitem}

\section{Megjegyzés a kidolgozáshoz}

A következőkben a 2022. júniusi vizsgára kiadott tételek kidolgozásai találhatóak.

Diák által kidolgozott verzió: amennyiben hibát talál, kérem jelezze a \textit{szilvia.csonka1996@gmail.com} email címre küldött üzenet formájában és lehetőség szerint igyekszem mielőbb javítani.

\section{A szóbeli és gyakorlati vizsgák követelményei}

\begin{center}
	\begin{longtable}{ | p{0.3\textwidth} | p{0.7\textwidth} | }
		
		\hline
		\multicolumn{2}{|c|}{\textbf{A szóbeli (elméleti) vizsga adatai}}
		\\ \hline
		
		\hline
		\centering{Szakképesítés megnevezése}
		& 
		4 0213 01 Képző- és iparművészeti munkatárs (02134002 Festő)
		\\ \hline
		
		\centering{A vizsgafeladat megnevezése}
		&
		Szakmai elmélet vizsgafeladat (A-B)
		\\\hline
		
		\centering{A vizsgafeladat ismertetése}
		&
		Szóbeli összefoglaló a programkövetelmény alapján összeállított, központi tételsorból húzott tétel A) és B) témaköreiből.
		\\\hline
		
		\centering{Vizsgafeladat időtartama}
		&
		45 perc\newline
		 - Felkészüési idő: 30 perc,\newline
		 - Válaszadási idő: 15 perc
		\\\hline
		
		\centering{A vizsgafeladat értékelési súlyaránya}
		&
		\textbf{30\%}
		\\\hline
		
		\centering{A tételsor a következő tantárgyak témaköreiből kerül összeállításra}
		&
		\textbf{A) Művészettörténet},\newline
		\textbf{B) Festő szakmai elmélet}.
		\\\hline
		
		\centering{Tételek tartalma}
		&
		A központilag összeállított vizsgakérdések a művészettörténeti és festő szakmai elméleti témaköröket tartalmazzák.
		\\\hline
		
		\centering{A tételekhez használható segédeszköz}
		&
		A képző intézmény által összeállított maximum 2-2 oldalas, feliratok nélküli képmelléklet.
		\\\hline		
	\end{longtable}
\end{center}

\begin{center}
	\begin{longtable}{ | p{0.3\textwidth} | p{0.7\textwidth} | }
		
		\hline
		\multicolumn{2}{|c|}{\textbf{A gyakorlati vizsga adatai}}
		\\ \hline
		
		\hline
		\centering{Gyakorlati vizsgatevékenység hivatalos ismertetése}
		& 
		\textbf{A) Szakmai portfólió} \newline
		\textbf{B) Festő vizsgamunka bemutatása}
		\\ \hline
		
		\centering{A vizsgafeladat időtartama}
		&
		A) és B) vizsgatevékenységre összesen: \textbf{30 perc}
		(felkészülési idő nincs).
		\\\hline
		
		\centering{Értékelési súlyaránya}
		&
		Az A) és B) vizsgatevékenységre összesen:\textbf{ 70\%}.
		\\\hline
		
		\centering{Vizsgafeladat ismertetése}
		&
		 Az A) Szakmai portfólió és a B) Festő vizsgamunka egymást követő bemutatása, melyeket egybefüggő tevékenységként kell megszervezni.
		\\\hline
		
		\centering{A) A Szakmai portfólió bemutatása}
		&
		A képzési időben végigvitt szakmai gyakorlati tevékenység bemutatása a képekkel és rövid szövegekkel megszerkesztett \textbf{portfólió prezentációjával}. Rajzi, plasztikai előtanulmányokat (minimum 6 db különböző munka), kreatív és szakmai feladatokat tartalmazó, együttesen minimum 25 oldalas digitális prezentáció.
		\\\hline
		
		\centering{A Szakmai portfólió anyagának és bemutatásának értékelése}
		&
		A vizsgázó a Gyakorlati vizsgán az előzetesen leadott digitális portfólióját mutatja be.\textbf{ Értelmezi a munkáit} és \textbf{ismerteti a munkáját irányító gondolati, technikai hátteret}. A vizsgabizottság a vizsgázó által elkészített \textbf{szakmai anyag együttesét, a megoldások színvonalát és a bemutató alkalmával mérhető alkotói gondolkodásmód} meglétét, fejlettségét, a verbális előadást és a szakmai fogalmak pontos használatát együttesen értékeli.
		\\\hline
		
		\centering{B) Szakmai vizsgamunka (vizsgaremek) bemutatása és prezentáció követelményei és annak értékelése}
		&
		A szakmai vizsgamunka készítése olyan szakmai tevékenység, amely önmagában alkalmas arra, hogy a vizsgázó beszámolhasson a tanulmányai során megszerzett ismeretek és képességek valamelyik komplex halmazából. A vizsgamunka a kész vizsgatárgyon kívül tartalmazzák a készítési folyamat leírását és dokumentációját nyomtatott és elektronikus prezentáció formájában (tervek, vázlatok, munkafázisok).
		\\\hline
		
	\end{longtable}
\end{center}


\cleardoublepage