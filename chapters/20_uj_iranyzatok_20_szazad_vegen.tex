\chapter{Újabb képzőművészeti műfajok a 20. század második felében és kortárs alkotók} % Introduction
\label{ch:20_uj_iranyzatok_20_szazad_vegen}

\section{Újabb képzőművészeti műfajok, irányzatok a 20. század második felében}

\begin{center}
	\begin{longtable}{ | p{0.25\textwidth} | p{0.75\textwidth} | }
		
		\hline
		\multicolumn{2}{|c|}{\textbf{A tétel adatai}}
		\\ \hline
		\hline
		
		\centering{Tétel teljes címe}
		&
		Mutassa be a 20. század második felében megjelenő újabb képzőművészeti műfajokat, irányzatokat!
		\\ \hline
		
		\centering{Jegyzetek}
		&
		\begin{compactitem}
			\item A korszak új művészeti megnyilvánulásai Európában és az Egyesült Államokban és Magyarországon.
			\item A konceptuális művészet törekvései, fluxus, performance, akcióművészet, az installáció és médiaművészet alkotásai, művészei.
		\end{compactitem}
		\\\hline
		
	\end{longtable}
\end{center}

\cleardoublepage


\section{Kortárs művészek és új technikai lehetőségek a kortárs festészetben}

\begin{center}
	\begin{longtable}{ | p{0.25\textwidth} | p{0.75\textwidth} | }
		
		\hline
		\multicolumn{2}{|c|}{\textbf{A tétel adatai}}
		\\ \hline
		\hline
		
		\centering{Tétel teljes címe}
		&
		Soroljon fel legalább 3 kortárs alkotót, akik meghatározóak a jelenkori magyar festészetben! Beszéljen művészetükről, festészetük sajátosságairól! Melyek azok az új technikai lehetőségek, eljárások, amelyeket a kortárs festészet meghatározó módon használ fel?
		\\ \hline
		
	\end{longtable}
\end{center}
