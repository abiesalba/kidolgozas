\chapter{Az itáliai reneszánsz művészet} % Introduction
\label{ch:8_reneszansz}

\section{Az itáliai reneszánsz építészet, szobrászat és festészet}

\begin{center}
	\begin{longtable}{ | p{0.25\textwidth} | p{0.75\textwidth} | }
		
		\hline
		\multicolumn{2}{|c|}{\textbf{A tétel adatai}}
		\\ \hline
		
		\hline
		Tétel teljes címe
		& 
		Mutassa be az itáliai reneszánsz építészet, szobrászat, festészet stílusjegyeit, jellemzőit, alkotóit és műveit!
		\\ \hline
		
		Jegyzetek &
		\begin{compactitem}
			\item Az olasz reneszánsz művészet kialakulása, korszakai, jellemzői, vívmányai.
			\item Az olasz reneszánsz épülettípusai, a szobrászat stílusjegyei az egyes korszakokban.
			\item A festészet műfajai, technikái; festők és alkotások.
		\end{compactitem}
		\\\hline
		
	\end{longtable}
\end{center}

\cleardoublepage


\section{Az itáliai reneszánsz freskó festészet}

\begin{center}
	\begin{longtable}{ | p{0.25\textwidth} | p{0.75\textwidth} | }
		
		\hline
		\multicolumn{2}{|c|}{\textbf{A tétel adatai}}
		\\ \hline
		
		\hline
		Tétel teljes címe 
		&
		Mi jellemzi az itáliai reneszánsz freskó festészetet? Mutassa be a freskókészítés munkamenetét! Milyen festékek és kötőanyagok használhatóak a freskó festészeti technikánál?
		\\ \hline
		
	\end{longtable}
\end{center}
