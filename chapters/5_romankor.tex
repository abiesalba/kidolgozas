\chapter{A románkori művészet} % Introduction
\label{ch:5_romankor}

\section{A románkori építészet, szobrászat és kézművesség}

\begin{center}
	\begin{longtable}{ | p{0.25\textwidth} | p{0.75\textwidth} | }
		
		\hline
		\multicolumn{2}{|c|}{\textbf{A tétel adatai}}
		\\ \hline
		
		\hline
		\centering{Tétel teljes címe}
		&
		Mutassa be a románkori építészet, szobrászat és kézművesség stílusjegyeit! Jellemezze a különböző műfajokhoz kapcsolódó alkotásokat, európai és hazai műemlékeinket!
		\\ \hline
		
		\centering{Jegyzetek}
		&
		\begin{compactitem}
			\item A románkor társadalma, hitvilága, építészete.
			\item A koraközépkori szobrászat, kódexfestészet és kézművesség jellemzői.
		\end{compactitem}
		\\\hline
		
	\end{longtable}
\end{center}

\cleardoublepage


\section{A románkori freskófestészet, kódexfestészet és alapozási technikák}

\begin{center}
	\begin{longtable}{ | p{0.25\textwidth} | p{0.75\textwidth} | }
		
		\hline
		\multicolumn{2}{|c|}{\textbf{A tétel adatai}}
		\\ \hline
		
		\hline
		\centering{Tétel teljes címe}
		&
		Milyen hasonlóságokat és különbségeket talál a román kori freskófestészet és a kódexfestészet között tartalmilag, formailag és technikailag? Mutassa be a középkorban elterjedt alapozási technikákat, pigmenteket, kötőanyagokat!
		\\ \hline
		
	\end{longtable}
\end{center}