\chapter{A klasszicizmus művészete} % Introduction
\label{ch:12_klasszicizmus}

\section{A kalsszicista építészet, szobrászat és festészet}

\begin{center}
	\begin{longtable}{ | p{0.25\textwidth} | p{0.75\textwidth} | }
		
		\hline
		\multicolumn{2}{|c|}{\textbf{A tétel adatai}}
		\\ \hline
		\hline
		
		\centering{Tétel teljes címe}
		&
		Mutassa be a klasszicista építészet, szobrászat és festészet stílusjegyeit, alkotásait, hazai műemlékeit!
		\\ \hline
		
		\centering{Jegyzetek}
		&
		\begin{compactitem}
			\item A klasszicizmus álltalános jellemzői, társadalmi, filozófiai háttere.
			\item Az építészet stílusjegyei (Fraqnciaország, Anglia, Magyarország).
			\item A szobrászat és a festészet stílusjegyei, mesterei (Franciaország és Magyarország).
		\end{compactitem}
		\\\hline
		
	\end{longtable}
\end{center}

\cleardoublepage


\section{Klasszikus megoldások a díszletfestészetben, a táblakép és a díszletfestő technika különbségei}
\begin{center}
	\begin{longtable}{ | p{0.25\textwidth} | p{0.75\textwidth} | }
		
		\hline
		\multicolumn{2}{|c|}{\textbf{A tétel adatai}}
		\\ \hline
		\hline
		
		\centering{Tétel teljes címe} 
		&
		Milyen technikai, ábrázolási illetve szerkesztési megoldásokat vett át a díszletfestészet a klasszikus festészettől? Melyek a legjellemzőbb különbségek a táblakép és a díszletfestő technikák között?
		\\ \hline
		
	\end{longtable}
\end{center}
