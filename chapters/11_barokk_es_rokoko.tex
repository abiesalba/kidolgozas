\chapter{A barokk és rokokó festészete} % Introduction
\label{ch:11_barokk_es_rokoko}

\section{A barokk és rokokó festészet európában}

\begin{center}
	\begin{longtable}{ | p{0.25\textwidth} | p{0.75\textwidth} | }
		
		\hline
		\multicolumn{2}{|c|}{\textbf{A tétel adatai}}
		\\ \hline
		\hline
		
		\centering{Tétel teljes címe}
		& 
		Mutassa be az egyes európai országok barokk és rokokó festészetének külböző stílusjegyeit és alkotásait!
		\\ \hline
		
		\centering{Jegyzetek}
		&
		\begin{compactitem}
			\item A barokk festészet álltalános jellemzői, szerepe.
			\item A katolikus Itália, Spanyolország, Flandria és polgári Hollandia festészete, mesterei.
			\item A magyarországi barokk festészet példái.
		\end{compactitem}
		\\\hline
		
	\end{longtable}
\end{center}

\cleardoublepage


\section{A rembrandti festészet jellemzői és olajfestés technikája}

\begin{center}
	\begin{longtable}{ | p{0.25\textwidth} | p{0.75\textwidth} | }
		
		\hline
		\multicolumn{2}{|c|}{\textbf{A tétel adatai}}
		\\ \hline
		
		\hline
		\centering{Tétel teljes címe}
		&
		Ismertese Rembrandt egy alkotásán keresztül a rembrandti festészet jellemzőit! Milyen olajok és oldószerek használhatóak az olajfestés technikájához?
		\\ \hline
		
	\end{longtable}
\end{center}
